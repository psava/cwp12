\section{Examples}
In order to analyze the backscattered energy I use a simple two layers model. I then 
use two cases: constant density and hight  velocity constrast, constant velocity and density
contrast.  

For each backscatering case I analyze the behaviour of the backscatering energy for different
velocity errors (from -25\% to +25\% in intervals of 5\%). By considering the hard interface in 
the density model we have the ability to turn off and on the back scatered energy in the receiver
 and source wavefields.

To exemplify this workflow we can consider the following example:
\begin{enumerate}
\item Propagate source to model $V(\bf{x}), \rho(\bf{x})$ to obtain $\US=\US^b +\US^n$
\item Propagate source to model $V(\bf{x}), \rho_o(\bf{x})$, where
$\rho_o$ is constant, to obtain $\US^n$.
\item Obtain $\US^b= \US -\US^n$.
\end{enumerate}

The same workflow applies to the receiver wavefield. By doing this we have the ability to separate all 
the cross-correlation cases that contributes to an image $R(\bf{x})$.

\subsection{Understanding the backscattered energy.}

In order to get an understanding of the phenomena, I do a very simple 2D example. I consider
a constant velocity medium as shown in figure \rfn{vel05} and, a two layered density model in figure \rfn{den05}. The idea of having the 
hard interface in the density model is that allow us to split in an easy way the cases shown in equation \ref{eq:cases}. 
The experiment is composed by one shot with a receiver array in the complete surface. The conventional image
given by equation ~\ref{eq:IC} is shown in figure ~\rfn{img05_ref}.




\inputdir{flat2}

\multiplot{3}{vel05,den05,img05_ref}{width=0.4\textwidth}{Constant velocit model of 2.2km/s (a), density model (b) 
and retreived image (c).}

\multiplot*{4}{img05_nn,img05_nb,img05_bn,img05_bb}{width=0.4\textwidth}{Understanding the backscattered energy by 
splitting $R(\bf{x})$ shown in figure ~\ref{fig:img05_ref} in: $R(\bf{x})^{nn}$ (a), $R(\bf{x})^{nb}$ (b), $R(\bf{x})^{bn}$ (c) and
$R(\bf{x})^{bb}$ (d).}