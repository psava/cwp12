\begin{abstract}
RTM backscattered events are produced by the cross-correlation between waves reflected from 
sharp interfaces (as top of salt). These events along with head waves and diving waves produces
the so-called RTM artifacts. Commonly these events are seen as a drawback of the RTM method.
Many strategies have been developed to filter them out of the conventional image. The filtering
step is necessary since the backscattered events are not correlated with the geology. 
We perform an interpretation of the RTM backscattered energy in conventional and extended
images. The analysis of these events shows that they contain important information
about the synchronization between wavefields. These events show kinematic sensitivity to 
velocity error. An optimum model in salt environment geological
settings exhibit a maximum energy of the backscattered events in the image, therefore
the velocity model and interfaces interpretation should be carried having in mind
this criterion. The analysis of RTM extended images can be used as a quality control tool 
to constrain salt models and sediment velocity.
\end{abstract}
