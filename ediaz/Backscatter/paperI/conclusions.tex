\section{Conclusions}

RTM backscattered events maps to zero lag in the extended images when the
velocity is correct. This means that the reflected
receiver wavefield travels in perfect synchronization with the source wavefield and
vice versa.

We demonstrate that the RTM backscattered energy is sensitive to the kinematics errors
on the velocity model. The backscattered energy in the final image should not be considered 
as an artifact or a drawback of the imaging method, rather the backscattered energy
 should be maximized in the image to ensure an optimum velocity 
model. The analysis backscattered energy on extended images provides a defined criteria on models with
salt interfaces (i.e Gulf of Mexico).
 Further tests are needed to develop automated velocity model update based on the maximization of 
the RTM backscattered energy on the CIC.

Tests on the Sigsbee using the correct model shows that some energy could be mapped away from zero space-lag when
we use RTM. This observation should be taken into account in the design of the objective 
functions in Migration Velocity Analysis (MVA) using the information from primary reflections.
