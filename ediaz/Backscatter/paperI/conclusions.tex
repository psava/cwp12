\section{Conclusions}

RTM backscattered events maps to zero lag in the extended images when the
velocity is correct. This means that the reflected
receiver wavefield travels in perfect synchronization with the source wavefield and
vice versa. We demonstrate that the RTM backscattered energy is sensitive to kinematics errors
in the velocity model. The backscattered energy in the final image should not be considered 
as an artifact or a drawback of the imaging method; rather, the backscattered energy
 should be maximized in the image in order to ensure an optimum velocity 
model. The analysis backscattered energy on extended images provide a defined criterion to update 
velocity models with salt interfaces (i.e in the Gulf of Mexico).
 Further tests are needed to develop automated velocity model update based on the maximization of 
the RTM backscattered energy.

Our analysis also shows that some energy is mapped away from zero space-lag when
we use RTM. This observation suggest that backscattered events should be taken into account in the design 
of objective functions in wavefield tomography using primary reflections information.
