\section{Introduction}

Reverse time migration (RTM) is not a new imaging technique ~\citep{baysal:1514, whitmore:382, GPR:GPR413}.
However, it was not until the late 1990’s, and mainly the 2000’s that computational
 advances allowed the geophysical community to use this technology for exploratory
3D surveys. In general, and especially in complex geological settings, RTM produces better 
images than other methods. Imaging methods such as Kirchhoff migration and one-way equation
 migration are based on approximate solutions to the wave equation. Kirchhoff migration,
 a high frequency asymptotic solution to the wave equation, becomes unstable for complex velocity models.
 This technique also fails to easy handle multipathing and typically creates the images based on a 
single travel-time arrival (e.g. most energetic or first arrival). Other methods based on approximations to the wave
 equation, such as phase shift migration ~\citep{gazdag:1342}, rely on a v(z) earth model and further
 approximations are needed to account for lateral variations~\citep{gazdag:124}.
 In addition to earth model considerations, one-way wave equation migration propagate the wavefields in 
either the upward or the downward direction; this approximation becomes inexact when the waves 
propagate horizontally. Therefore, this technique fails to properly handle overturning
waves and reflections from steep-dip structures. RTM's propagation engine, a two-way wave equation, 
makes this imaging method robust and accurate because it honors the kinematics of 
the wave phenomena by allowing waves to propagate in all directions regardless of
 the velocity model or the direction of propagation. This method also takes into account,
 in a natural way, multipathing and  reflections from steep dips.

A striking characteristic of RTM is the presence of low wavenumber events in the image that
 are not correlated with the geology. The two-way wave equation simulates scattered waves in all 
directions. Therefore, the imaging condition produces new events not observed in 
other imaging methods that correspond to the cross-correlation between diving waves, head waves 
and backscattered waves. The cross-correlation between the backscattered waves is more visible in
presence of sharp boundaries (e.g. the top of salt) which produces strong events that mask the image of the 
earth reflectivity above the salt. The backscattered events are considered as noise and are normally filtered in order
to get the image of earth reflectivity.

The seismic industry has dedicated effort and time developing algorithms and strategies
to filter out the backscattered energy from the image. We can classify the filtering
 approaches in two general families: pre-imaging condition and post-imaging condition. 

The pre-imaging condition family modify the wavefields (either by modeling or wavefield decomposition)
 in such a way that the backscattered events do not form during the imaging process.
One strategy in the pre-imaging condition category is wavefield decomposition
~\citep[]{liu:S29,fei:3130}. In this method, the source and receiver wavefield are decomposed in upgoing
and downgoing directions. 
In the imaging step, we cross-correlate only the wavefields that propagate
in opposite directions producing an image which corresponds to the geology. The cross-correlation between wavefields traveling
in parallel directions is discarded because produce events that obstructs the geology. Other pre-imaging 
condition approaches are performed by modifying  the wave equation to attenuate the reflections coming from 
sharp interfaces~\citep{fletcher:2049}. A similar method applicable to post-stack migration 
uses impedance matching at sharp interfaces~\citep{baysal:132}. 

In the post-imaging family, the artifacts are attenuated by filtering. These filtering approaches
 are considerably cheaper because they operate in the image space and not on the wavefields. A straightforward
 approach is to apply a Laplacian operator to the image~\citep{youn:246}; this operator
 acts as a high pass filter and is effective because the backscattered events have a strong low wavenumber
component. A second strategy is a signal/noise  separation by least squares filtering. In
 this case the signal is defined as the reflectivity and the noise is the backscattered energy
~\citep{guitton:S19}. Finally, extended imaging conditions~\citep{rickett:883,sava:S209,GPR:GPR888} provide information
 about the wavefield similarity for different space and/or time lags and can also be used to discriminate
the backscattered energy.~\cite{kaelin:3125} take advantage of the way backscattered events appear in 
time-lag gathers. The backscattered events map toward zero time-lag when a correct velocity model is used for imaging,
 whereas the primary reflections map within a limited slope range constrained by the velocity model. This difference in
 slope allows us to design 2D filters that preserve events within the primaries reflections range and
attenuate the backscattered energy.

In this report we analyze the information carried by the backscattered energy in the extended
 images. We show that the backscattered waves provide important information about the 
synchronization between the reconstructed wavefields in the subsurface, i.e. an image obtained with a correct velocity model shows maximum backscattered
energy. The presence of backscattered energy in the image not only depends on the interpretation
of the sharp interface but also on the velocity above it. We analyze the mapping patterns of the backscattered
events in the extended images using wavefield decomposition approaches and conclude that backscattered energy
is sensitive to the velocity model accuracy and therefore should be included as a source
of information to migration velocity analysis (MVA). Counter to common practice, we assert that
backscattering artifacts should be enhanced during RTM to constrain the velocity models,
and they should only be removed in the last stage of imaging.



% Talk about Reverse Time Migration
% Talk about different techniques to attenuate artifacts
% Introduce the paper objective
%~\cite{claerbout:467} 
%~\cite{baysal:1514}
%~\cite{whitmore:382} 
%~\cite{guitton:S19} 
%~\cite{kaelin:3125} 
%~\cite{fletcher:2049} 
%~\cite{sava:S209} 
%~\cite{fei:3130} 
%~\cite{GPR:GPR413} 
%~\cite{baysal:132}
