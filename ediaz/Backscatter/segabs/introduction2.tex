\section{Introduction}

Reverse time migration (RTM) is not a new imaging technique ~\citep{baysal:1514, whitmore:382, GPR:GPR413}.
However, it was not until the late 1990’s that computational
 advances allowed the geophysical community to use this technology for exploratory
3D surveys. In complex geological settings, RTM produces better 
images than other methods. 

A striking characteristic of RTM is the presence of low wavenumber events, uncorrelated with
the geology, in models
with sharp interfaces (e.g. salt intrusions). The seismic industry has dedicated effort and time developing algorithms and strategies
to filter out the backscattered energy from the image. We can classify the filtering
 approaches in two general families: pre-imaging condition and post-imaging condition. 

The pre-imaging condition family modifes the wavefields (either by modeling or by wavefield decomposition)
 such that the backscattered events do not form during the imaging process.
One strategy is wavefield decomposition~\citep[]{liu:S29,fei:3130}. In this method only 
 wavefields propagating in opposite directions are cross-correlated. The modeling approach modifes the wavefields by not
allowing waves to reflect during propagation. One way to achieve this is by introducing an absorbing
boundary condition in the wave equation at the top of salt~\citep{fletcher:2049}, or 
by impedance matching~\citep{baysal:132}. 


In the post-imaging family, the artifacts are attenuated by filtering, which is a 
 considerably cheaper process because operates in the image space and not on the much larger and more complex wavefields. A common
strategy is to apply a Laplacian filter to the image~\citep{youn:246}. Another option
is a signal/noise separation by least-squares filtering~\citep{guitton:S19}. One can also consider using
extended images~\citep{rickett:883,sava:S209,GPR:GPR888} for filtering. This method~\citep{kaelin:3125}
 takes advantage of the slope difference between primary and 
backscattered events in the time-lag gathers.

In this paper, we analyze the information carried by the backscattered energy in the extended
 images. We show that the backscattered waves provide important information about the 
synchronization between the reconstructed wavefields in the subsurface.
 The presence of backscattered energy in the image not only depends on the interpretation
of the sharp interface but also on the velocity above it. We analyze the mapping patterns of the backscattered
events in the extended images and conclude that backscattered energy
is sensitive to the velocity model accuracy and therefore should be used as a source
of information for migration velocity analysis (MVA). Counter to common practice, we assert that
backscattering artifacts should be enhanced during RTM to constrain the velocity models,
and they should only be removed in the last stage of imaging.



% Talk about Reverse Time Migration
% Talk about different techniques to attenuate artifacts
% Introduce the paper objective
%~\cite{claerbout:467} 
%~\cite{baysal:1514}
%~\cite{whitmore:382} 
%~\cite{guitton:S19} 
%~\cite{kaelin:3125} 
%~\cite{fletcher:2049} 
%~\cite{sava:S209} 
%~\cite{fei:3130} 
%~\cite{GPR:GPR413} 
%~\cite{baysal:132}
