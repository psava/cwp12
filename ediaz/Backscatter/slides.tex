

\title[]{Including RTM backscattered in VMB flow}
\subtitle{GPGN658 project }
\author[]{Esteban  D\'{i}az}
\institute{
Center for Wave Phenomena \\
Colorado School of Mines \\
ediazpan@mines.edu
}
\date{}
\logo{}

\def\big#1{\begin{center} \LARGE \textbf{#1} \end{center}}
\def\cen#1{\begin{center}        \textbf{#1} \end{center}}

% ------------------------------------------------------------
\mode<beamer> { \cwpcover }

% ------------------------------------------------------------
\inputdir{gpgn658}
% ------------------------------------------------------------
\begin{frame} \frametitle{Backscattered artifacts}
The idea behind it, and some intuitive thoughts:

\begin{itemize}
   \item The position of the ``artifacts'' should tell 
   something about the synchronization between wavefields.
   \item The energy of it should indicate a maximum synchronization 
   between wavefields.
\end{itemize}

\end{frame}
\cwpnote{}

% ------------------------------------------------------------
\begin{frame} \frametitle{Test experiments}

\begin{itemize}
   \item 3 velocity models: correct, 95\% correct, 112\% correct
   \item The hard interface is in the density model. 
\end{itemize}

\end{frame}
\cwpnote{}

% ------------------------------------------------------------
\begin{frame} \frametitle{Correct model} \plot{vel_den0}{width=\textwidth}{} \end{frame}
\begin{frame} \frametitle{Slow    model} \plot{vel_den1}{width=\textwidth}{} \end{frame}
\begin{frame} \frametitle{Fast    model} \plot{vel_den2}{width=\textwidth}{} \end{frame}

% ------------------------------------------------------------
\begin{frame} \frametitle{Correct model image} \plot{img0}{width=\textwidth}{} \end{frame}
\begin{frame} \frametitle{Slow model image   } \plot{img1}{width=\textwidth}{} \end{frame}
\begin{frame} \frametitle{Fast model image   } \plot{img2}{width=\textwidth}{} \end{frame}

% ------------------------------------------------------------
\begin{frame} \frametitle{Correct time-lag   } \plot{cit0}{height=0.8\textheight}{} \end{frame}
\begin{frame} \frametitle{Slow model time-lag} \plot{cit1}{height=0.8\textheight}{} \end{frame}
\begin{frame} \frametitle{Fast model time-lag} \plot{cit2}{height=0.8\textheight}{} \end{frame}



% ------------------------------------------------------------
\begin{frame} \frametitle{Image construction for correct model} \end{frame}
\begin{frame} \plot{WSI0-000}{width=\textwidth}{} \end{frame}
\begin{frame} \plot{WSI0-025}{width=\textwidth}{} \end{frame}
\begin{frame} \plot{WSI0-050}{width=\textwidth}{} \end{frame}
\begin{frame} \plot{WSI0-075}{width=\textwidth}{} \end{frame}
\begin{frame} \plot{WSI0-100}{width=\textwidth}{} \end{frame}
\begin{frame} \plot{WSI0-125}{width=\textwidth}{} \end{frame}
\begin{frame} \plot{WSI0-150}{width=\textwidth}{} \end{frame}
\begin{frame} \plot{WSI0-175}{width=\textwidth}{} \end{frame}
\begin{frame} \plot{WSI0-200}{width=\textwidth}{} \end{frame}
\begin{frame} \plot{WSI0-225}{width=\textwidth}{} \end{frame}
\begin{frame} \plot{WSI0-250}{width=\textwidth}{} \end{frame}
\begin{frame} \plot{WSI0-275}{width=\textwidth}{} \end{frame}
\begin{frame} \plot{WSI0-300}{width=\textwidth}{} \end{frame}
\begin{frame} \plot{WSI0-325}{width=\textwidth}{} \end{frame}
\begin{frame} \plot{WSI0-350}{width=\textwidth}{} \end{frame}
\begin{frame} \plot{WSI0-375}{width=\textwidth}{} \end{frame}
\begin{frame} \plot{WSI0-400}{width=\textwidth}{} \end{frame}
\begin{frame} \plot{WSI0-425}{width=\textwidth}{} \end{frame}
\begin{frame} \plot{WSI0-450}{width=\textwidth}{} \end{frame}
\begin{frame} \plot{WSI0-475}{width=\textwidth}{} \end{frame}
\begin{frame} \plot{WSI0-500}{width=\textwidth}{} \end{frame}
\begin{frame} \plot{WSI0-525}{width=\textwidth}{} \end{frame}
\begin{frame} \plot{WSI0-550}{width=\textwidth}{} \end{frame}
\begin{frame} \plot{WSI0-575}{width=\textwidth}{} \end{frame}
\begin{frame} \plot{WSI0-600}{width=\textwidth}{} \end{frame}


% ------------------------------------------------------------
\begin{frame} \frametitle{Image construction for slow model} \end{frame}
\begin{frame} \plot{WSI1-000}{width=\textwidth}{} \end{frame}
\begin{frame} \plot{WSI1-025}{width=\textwidth}{} \end{frame}
\begin{frame} \plot{WSI1-050}{width=\textwidth}{} \end{frame}
\begin{frame} \plot{WSI1-075}{width=\textwidth}{} \end{frame}
\begin{frame} \plot{WSI1-100}{width=\textwidth}{} \end{frame}
\begin{frame} \plot{WSI1-125}{width=\textwidth}{} \end{frame}
\begin{frame} \plot{WSI1-150}{width=\textwidth}{} \end{frame}
\begin{frame} \plot{WSI1-175}{width=\textwidth}{} \end{frame}
\begin{frame} \plot{WSI1-200}{width=\textwidth}{} \end{frame}
\begin{frame} \plot{WSI1-225}{width=\textwidth}{} \end{frame}
\begin{frame} \plot{WSI1-250}{width=\textwidth}{} \end{frame}
\begin{frame} \plot{WSI1-275}{width=\textwidth}{} \end{frame}
\begin{frame} \plot{WSI1-300}{width=\textwidth}{} \end{frame}
\begin{frame} \plot{WSI1-325}{width=\textwidth}{} \end{frame}
\begin{frame} \plot{WSI1-350}{width=\textwidth}{} \end{frame}
\begin{frame} \plot{WSI1-375}{width=\textwidth}{} \end{frame}
\begin{frame} \plot{WSI1-400}{width=\textwidth}{} \end{frame}
\begin{frame} \plot{WSI1-425}{width=\textwidth}{} \end{frame}
\begin{frame} \plot{WSI1-450}{width=\textwidth}{} \end{frame}
\begin{frame} \plot{WSI1-475}{width=\textwidth}{} \end{frame}
\begin{frame} \plot{WSI1-500}{width=\textwidth}{} \end{frame}
\begin{frame} \plot{WSI1-525}{width=\textwidth}{} \end{frame}
\begin{frame} \plot{WSI1-550}{width=\textwidth}{} \end{frame}
\begin{frame} \plot{WSI1-575}{width=\textwidth}{} \end{frame}
\begin{frame} \plot{WSI1-600}{width=\textwidth}{} \end{frame}



% ------------------------------------------------------------
\begin{frame} \frametitle{Image construction for fast model} \end{frame}
\begin{frame} \plot{WSI2-000}{width=\textwidth}{} \end{frame}
\begin{frame} \plot{WSI2-025}{width=\textwidth}{} \end{frame}
\begin{frame} \plot{WSI2-050}{width=\textwidth}{} \end{frame}
\begin{frame} \plot{WSI2-075}{width=\textwidth}{} \end{frame}
\begin{frame} \plot{WSI2-100}{width=\textwidth}{} \end{frame}
\begin{frame} \plot{WSI2-125}{width=\textwidth}{} \end{frame}
\begin{frame} \plot{WSI2-150}{width=\textwidth}{} \end{frame}
\begin{frame} \plot{WSI2-175}{width=\textwidth}{} \end{frame}
\begin{frame} \plot{WSI2-200}{width=\textwidth}{} \end{frame}
\begin{frame} \plot{WSI2-225}{width=\textwidth}{} \end{frame}
\begin{frame} \plot{WSI2-250}{width=\textwidth}{} \end{frame}
\begin{frame} \plot{WSI2-275}{width=\textwidth}{} \end{frame}
\begin{frame} \plot{WSI2-300}{width=\textwidth}{} \end{frame}
\begin{frame} \plot{WSI2-325}{width=\textwidth}{} \end{frame}
\begin{frame} \plot{WSI2-350}{width=\textwidth}{} \end{frame}
\begin{frame} \plot{WSI2-375}{width=\textwidth}{} \end{frame}
\begin{frame} \plot{WSI2-400}{width=\textwidth}{} \end{frame}
\begin{frame} \plot{WSI2-425}{width=\textwidth}{} \end{frame}
\begin{frame} \plot{WSI2-450}{width=\textwidth}{} \end{frame}
\begin{frame} \plot{WSI2-475}{width=\textwidth}{} \end{frame}
\begin{frame} \plot{WSI2-500}{width=\textwidth}{} \end{frame}
\begin{frame} \plot{WSI2-525}{width=\textwidth}{} \end{frame}
\begin{frame} \plot{WSI2-550}{width=\textwidth}{} \end{frame}
\begin{frame} \plot{WSI2-575}{width=\textwidth}{} \end{frame}
\begin{frame} \plot{WSI2-600}{width=\textwidth}{} \end{frame}






%-----------------------------------------------------------

\begin{frame}

Next step: separate backscattered energy from the primary event 
in the time-lag gathers.

I am thinking in two stratagies:
\begin{itemize}
\item F-K dip filter (Kaelin and Carvajal, 2011)
\item Calculate time-lag gathers with constant density and 
substract to the gathers from the variable density.
\end{itemize}
\end{frame}

