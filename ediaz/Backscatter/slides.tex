%% This file is automatically generated. Do not edit!
\documentclass[12pt]{cwpslides}

\usepackage{multicol}
\usepackage{color}
\usepackage{overpic}
\usepackage{listings}
\usepackage{amsmath}
\usepackage{amssymb}
\usepackage{amsbsy}

\begin{document}
\input{pcsmacros}

\title[]{Including RTM backscattered in VMB flow}
\subtitle{GPGN658 project }
\author[]{Esteban ``Toto'' D\'{i}az}
\institute{
Center for Wave Phenomena \\
Colorado School of Mines \\
ediazpan@mines.edu
}
\date{}
\logo{}

\def\big#1{\begin{center} \LARGE \textbf{#1} \end{center}}
\def\cen#1{\begin{center}        \textbf{#1} \end{center}}

% ------------------------------------------------------------
\mode<beamer> { \cwpcover }

% ------------------------------------------------------------
\inputdir{gpgn658}
% ------------------------------------------------------------

% ------------------------------------------------------------
\begin{frame} \frametitle{Backscattered artifacts}
The idea behind it, and some intuitive thoughts:

\begin{itemize}
   \item The position of the ``artifacts'' should tell 
   something about the syncronization between wavefields.
   \item The energy of it should indicate a maximum syncronization 
   between wavefields.
\end{itemize}

\end{frame}
\cwpnote{}

% ------------------------------------------------------------
\begin{frame} \frametitle{Test experiments}

\begin{itemize}
   \item 3 velocity models: correct, 95\% correct, 112\% correct
   \item The hard interface is in the density model. 
\end{itemize}

\end{frame}
\cwpnote{}

% ------------------------------------------------------------
\begin{frame} \frametitle{Correct model} \plot{vel_den0}{width=\textwidth}{} \end{frame}
\begin{frame} \frametitle{Slow    model} \plot{vel_den1}{width=\textwidth}{} \end{frame}
\begin{frame} \frametitle{Fast    model} \plot{vel_den2}{width=\textwidth}{} \end{frame}

% ------------------------------------------------------------
\begin{frame} \frametitle{Correct model image} \plot{img0}{width=\textwidth}{} \end{frame}
\begin{frame} \frametitle{Slow model image   } \plot{img1}{width=\textwidth}{} \end{frame}
\begin{frame} \frametitle{Fast model image   } \plot{img2}{width=\textwidth}{} \end{frame}

% ------------------------------------------------------------
\begin{frame} \frametitle{Correct time-lag   } \plot{cit0}{width=\textwidth}{} \end{frame}
\begin{frame} \frametitle{Slow model time-lag} \plot{cit1}{width=\textwidth}{} \end{frame}
\begin{frame} \frametitle{Fast model time-lag} \plot{cit2}{width=\textwidth}{} \end{frame}

% ------------------------------------------------------------
\begin{frame} \plot{WSI0-000}{width=\textwidth}{} \end{frame}
\begin{frame} \plot{WSI0-025}{width=\textwidth}{} \end{frame}
\begin{frame} \plot{WSI0-050}{width=\textwidth}{} \end{frame}
\begin{frame} \plot{WSI0-075}{width=\textwidth}{} \end{frame}
\begin{frame} \plot{WSI0-100}{width=\textwidth}{} \end{frame}
\begin{frame} \plot{WSI0-125}{width=\textwidth}{} \end{frame}
\begin{frame} \plot{WSI0-150}{width=\textwidth}{} \end{frame}
\begin{frame} \plot{WSI0-175}{width=\textwidth}{} \end{frame}
\begin{frame} \plot{WSI0-200}{width=\textwidth}{} \end{frame}
\begin{frame} \plot{WSI0-225}{width=\textwidth}{} \end{frame}
\begin{frame} \plot{WSI0-250}{width=\textwidth}{} \end{frame}
\begin{frame} \plot{WSI0-275}{width=\textwidth}{} \end{frame}
\begin{frame} \plot{WSI0-300}{width=\textwidth}{} \end{frame}
\begin{frame} \plot{WSI0-325}{width=\textwidth}{} \end{frame}
\begin{frame} \plot{WSI0-350}{width=\textwidth}{} \end{frame}
\begin{frame} \plot{WSI0-375}{width=\textwidth}{} \end{frame}
\begin{frame} \plot{WSI0-400}{width=\textwidth}{} \end{frame}
\begin{frame} \plot{WSI0-425}{width=\textwidth}{} \end{frame}
\begin{frame} \plot{WSI0-450}{width=\textwidth}{} \end{frame}
\begin{frame} \plot{WSI0-475}{width=\textwidth}{} \end{frame}
\begin{frame} \plot{WSI0-500}{width=\textwidth}{} \end{frame}
\begin{frame} \plot{WSI0-525}{width=\textwidth}{} \end{frame}
\begin{frame} \plot{WSI0-550}{width=\textwidth}{} \end{frame}
\begin{frame} \plot{WSI0-575}{width=\textwidth}{} \end{frame}
\begin{frame} \plot{WSI0-600}{width=\textwidth}{} \end{frame}

% ------------------------------------------------------------

% ------------------------------------------------------------
\begin{frame}

  \big{\url{http://www.ahay.org}}
  
  \vfill

  \big{\texttt{\$RSFSRC/book/rsf/school}}

  \vfill
  
\end{frame}


\end{document}
