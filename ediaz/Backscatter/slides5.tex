\bibliographystyle{seg}
\input{./Share/pcsmacros}

\bibliography{SEG}

\title[]{Understanding the backscattered energy in RTM: noise or signal?}
\subtitle{}
\author[]{Esteban  D\'{i}az}
\institute{
Center for Wave Phenomena \\
Colorado School of Mines \\
ediazpan@mines.edu
}
\date{}
\logo{}

\def\big#1{\begin{center} \LARGE \textbf{#1} \end{center}}
\def\cen#1{\begin{center}        \textbf{#1} \end{center}}

% ------------------------------------------------------------
\mode<beamer> { \cwpcover }

% ------------------------------------------------------------


\begin{frame}
	\begin{itemize}
		\item What it is?  
		\item How is it formed?
		\item Why is it considered as noise?
		\item Why would I say is signal? 
		\item Can we use it for model update?
	\end{itemize}
\end{frame}




\begin{frame}
   What have we learn so far from this events?
\end{frame}

\begin{frame}
Modeling domain approach, without.
\end{frame}

\begin{frame}
With. From ~\cite{fletcher:2049}
\end{frame}


\begin{frame}
	\begin{itemize}
		\item As we saw before, the backscattered energy masks the geology.
		\item But... If this energy is strong and is spread over the image means we did great with the model!
	\end{itemize}
\end{frame}




\inputdir{flat}
\begin{frame} \big{Slow velocity (-20\% error) movie} \end{frame}
%-----------------------
 \begin{frame} 
 \begin{columns} 
    \column{0.5\textwidth}
      \plot{imgs-b_r-b01-000}{width=1\textwidth}{}
      \plot{wts-b01-000}{width=1\textwidth}{}
    \column{0.5\textwidth}
      \plot{cross_corrs-b_r-b01-000}{width=1\textwidth}{}
      \plot{wtr-b01-000}{width=1\textwidth}{}
 \end{columns}
\end{frame}
%-----------------------
 \begin{frame} 
 \begin{columns} 
    \column{0.5\textwidth}
      \plot{imgs-b_r-b01-025}{width=1\textwidth}{}
      \plot{wts-b01-025}{width=1\textwidth}{}
    \column{0.5\textwidth}
      \plot{cross_corrs-b_r-b01-025}{width=1\textwidth}{}
      \plot{wtr-b01-025}{width=1\textwidth}{}
 \end{columns}
\end{frame}
%-----------------------
 \begin{frame} 
 \begin{columns} 
    \column{0.5\textwidth}
      \plot{imgs-b_r-b01-050}{width=1\textwidth}{}
      \plot{wts-b01-050}{width=1\textwidth}{}
    \column{0.5\textwidth}
      \plot{cross_corrs-b_r-b01-050}{width=1\textwidth}{}
      \plot{wtr-b01-050}{width=1\textwidth}{}
 \end{columns}
\end{frame}
%-----------------------
 \begin{frame} 
 \begin{columns} 
    \column{0.5\textwidth}
      \plot{imgs-b_r-b01-075}{width=1\textwidth}{}
      \plot{wts-b01-075}{width=1\textwidth}{}
    \column{0.5\textwidth}
      \plot{cross_corrs-b_r-b01-075}{width=1\textwidth}{}
      \plot{wtr-b01-075}{width=1\textwidth}{}
 \end{columns}
\end{frame}
%-----------------------
 \begin{frame} 
 \begin{columns} 
    \column{0.5\textwidth}
      \plot{imgs-b_r-b01-100}{width=1\textwidth}{}
      \plot{wts-b01-100}{width=1\textwidth}{}
    \column{0.5\textwidth}
      \plot{cross_corrs-b_r-b01-100}{width=1\textwidth}{}
      \plot{wtr-b01-100}{width=1\textwidth}{}
 \end{columns}
\end{frame}
%-----------------------
 \begin{frame} 
 \begin{columns} 
    \column{0.5\textwidth}
      \plot{imgs-b_r-b01-125}{width=1\textwidth}{}
      \plot{wts-b01-125}{width=1\textwidth}{}
    \column{0.5\textwidth}
      \plot{cross_corrs-b_r-b01-125}{width=1\textwidth}{}
      \plot{wtr-b01-125}{width=1\textwidth}{}
 \end{columns}
\end{frame}
%-----------------------
 \begin{frame} 
 \begin{columns} 
    \column{0.5\textwidth}
      \plot{imgs-b_r-b01-150}{width=1\textwidth}{}
      \plot{wts-b01-150}{width=1\textwidth}{}
    \column{0.5\textwidth}
      \plot{cross_corrs-b_r-b01-150}{width=1\textwidth}{}
      \plot{wtr-b01-150}{width=1\textwidth}{}
 \end{columns}
\end{frame}
%-----------------------
 \begin{frame} 
 \begin{columns} 
    \column{0.5\textwidth}
      \plot{imgs-b_r-b01-175}{width=1\textwidth}{}
      \plot{wts-b01-175}{width=1\textwidth}{}
    \column{0.5\textwidth}
      \plot{cross_corrs-b_r-b01-175}{width=1\textwidth}{}
      \plot{wtr-b01-175}{width=1\textwidth}{}
 \end{columns}
\end{frame}
%-----------------------
 \begin{frame} 
 \begin{columns} 
    \column{0.5\textwidth}
      \plot{imgs-b_r-b01-200}{width=1\textwidth}{}
      \plot{wts-b01-200}{width=1\textwidth}{}
    \column{0.5\textwidth}
      \plot{cross_corrs-b_r-b01-200}{width=1\textwidth}{}
      \plot{wtr-b01-200}{width=1\textwidth}{}
 \end{columns}
\end{frame}
%-----------------------
 \begin{frame} 
 \begin{columns} 
    \column{0.5\textwidth}
      \plot{imgs-b_r-b01-225}{width=1\textwidth}{}
      \plot{wts-b01-225}{width=1\textwidth}{}
    \column{0.5\textwidth}
      \plot{cross_corrs-b_r-b01-225}{width=1\textwidth}{}
      \plot{wtr-b01-225}{width=1\textwidth}{}
 \end{columns}
\end{frame}
%-----------------------
 \begin{frame} 
 \begin{columns} 
    \column{0.5\textwidth}
      \plot{imgs-b_r-b01-250}{width=1\textwidth}{}
      \plot{wts-b01-250}{width=1\textwidth}{}
    \column{0.5\textwidth}
      \plot{cross_corrs-b_r-b01-250}{width=1\textwidth}{}
      \plot{wtr-b01-250}{width=1\textwidth}{}
 \end{columns}
\end{frame}
%-----------------------
 \begin{frame} 
 \begin{columns} 
    \column{0.5\textwidth}
      \plot{imgs-b_r-b01-275}{width=1\textwidth}{}
      \plot{wts-b01-275}{width=1\textwidth}{}
    \column{0.5\textwidth}
      \plot{cross_corrs-b_r-b01-275}{width=1\textwidth}{}
      \plot{wtr-b01-275}{width=1\textwidth}{}
 \end{columns}
\end{frame}
%-----------------------
 \begin{frame} 
 \begin{columns} 
    \column{0.5\textwidth}
      \plot{imgs-b_r-b01-300}{width=1\textwidth}{}
      \plot{wts-b01-300}{width=1\textwidth}{}
    \column{0.5\textwidth}
      \plot{cross_corrs-b_r-b01-300}{width=1\textwidth}{}
      \plot{wtr-b01-300}{width=1\textwidth}{}
 \end{columns}
\end{frame}
%-----------------------
 \begin{frame} 
 \begin{columns} 
    \column{0.5\textwidth}
      \plot{imgs-b_r-b01-325}{width=1\textwidth}{}
      \plot{wts-b01-325}{width=1\textwidth}{}
    \column{0.5\textwidth}
      \plot{cross_corrs-b_r-b01-325}{width=1\textwidth}{}
      \plot{wtr-b01-325}{width=1\textwidth}{}
 \end{columns}
\end{frame}
%-----------------------
 \begin{frame} 
 \begin{columns} 
    \column{0.5\textwidth}
      \plot{imgs-b_r-b01-350}{width=1\textwidth}{}
      \plot{wts-b01-350}{width=1\textwidth}{}
    \column{0.5\textwidth}
      \plot{cross_corrs-b_r-b01-350}{width=1\textwidth}{}
      \plot{wtr-b01-350}{width=1\textwidth}{}
 \end{columns}
\end{frame}
%-----------------------
 \begin{frame} 
 \begin{columns} 
    \column{0.5\textwidth}
      \plot{imgs-b_r-b01-375}{width=1\textwidth}{}
      \plot{wts-b01-375}{width=1\textwidth}{}
    \column{0.5\textwidth}
      \plot{cross_corrs-b_r-b01-375}{width=1\textwidth}{}
      \plot{wtr-b01-375}{width=1\textwidth}{}
 \end{columns}
\end{frame}
%-----------------------
 \begin{frame} 
 \begin{columns} 
    \column{0.5\textwidth}
      \plot{imgs-b_r-b01-400}{width=1\textwidth}{}
      \plot{wts-b01-400}{width=1\textwidth}{}
    \column{0.5\textwidth}
      \plot{cross_corrs-b_r-b01-400}{width=1\textwidth}{}
      \plot{wtr-b01-400}{width=1\textwidth}{}
 \end{columns}
\end{frame}
%-----------------------
 \begin{frame} 
 \begin{columns} 
    \column{0.5\textwidth}
      \plot{imgs-b_r-b01-425}{width=1\textwidth}{}
      \plot{wts-b01-425}{width=1\textwidth}{}
    \column{0.5\textwidth}
      \plot{cross_corrs-b_r-b01-425}{width=1\textwidth}{}
      \plot{wtr-b01-425}{width=1\textwidth}{}
 \end{columns}
\end{frame}
%-----------------------
 \begin{frame} 
 \begin{columns} 
    \column{0.5\textwidth}
      \plot{imgs-b_r-b01-450}{width=1\textwidth}{}
      \plot{wts-b01-450}{width=1\textwidth}{}
    \column{0.5\textwidth}
      \plot{cross_corrs-b_r-b01-450}{width=1\textwidth}{}
      \plot{wtr-b01-450}{width=1\textwidth}{}
 \end{columns}
\end{frame}
%-----------------------
 \begin{frame} 
 \begin{columns} 
    \column{0.5\textwidth}
      \plot{imgs-b_r-b01-475}{width=1\textwidth}{}
      \plot{wts-b01-475}{width=1\textwidth}{}
    \column{0.5\textwidth}
      \plot{cross_corrs-b_r-b01-475}{width=1\textwidth}{}
      \plot{wtr-b01-475}{width=1\textwidth}{}
 \end{columns}
\end{frame}
%-----------------------
 \begin{frame} 
 \begin{columns} 
    \column{0.5\textwidth}
      \plot{imgs-b_r-b01-500}{width=1\textwidth}{}
      \plot{wts-b01-500}{width=1\textwidth}{}
    \column{0.5\textwidth}
      \plot{cross_corrs-b_r-b01-500}{width=1\textwidth}{}
      \plot{wtr-b01-500}{width=1\textwidth}{}
 \end{columns}
\end{frame}
%-----------------------
 \begin{frame} 
 \begin{columns} 
    \column{0.5\textwidth}
      \plot{imgs-b_r-b01-525}{width=1\textwidth}{}
      \plot{wts-b01-525}{width=1\textwidth}{}
    \column{0.5\textwidth}
      \plot{cross_corrs-b_r-b01-525}{width=1\textwidth}{}
      \plot{wtr-b01-525}{width=1\textwidth}{}
 \end{columns}
\end{frame}
%-----------------------
 \begin{frame} 
 \begin{columns} 
    \column{0.5\textwidth}
      \plot{imgs-b_r-b01-550}{width=1\textwidth}{}
      \plot{wts-b01-550}{width=1\textwidth}{}
    \column{0.5\textwidth}
      \plot{cross_corrs-b_r-b01-550}{width=1\textwidth}{}
      \plot{wtr-b01-550}{width=1\textwidth}{}
 \end{columns}
\end{frame}
%-----------------------
 \begin{frame} 
 \begin{columns} 
    \column{0.5\textwidth}
      \plot{imgs-b_r-b01-575}{width=1\textwidth}{}
      \plot{wts-b01-575}{width=1\textwidth}{}
    \column{0.5\textwidth}
      \plot{cross_corrs-b_r-b01-575}{width=1\textwidth}{}
      \plot{wtr-b01-575}{width=1\textwidth}{}
 \end{columns}
\end{frame}
%-----------------------
 \begin{frame} 
 \begin{columns} 
    \column{0.5\textwidth}
      \plot{imgs-b_r-b01-600}{width=1\textwidth}{}
      \plot{wts-b01-600}{width=1\textwidth}{}
    \column{0.5\textwidth}
      \plot{cross_corrs-b_r-b01-600}{width=1\textwidth}{}
      \plot{wtr-b01-600}{width=1\textwidth}{}
 \end{columns}
\end{frame}

\begin{frame} \big{Correct velocity movie} \end{frame}
%-----------------------
 \begin{frame} 
 \begin{columns} 
    \column{0.5\textwidth}
      \plot{wts-b05-000}{width=1\textwidth}{}
      \plot{wtr-b05-000}{width=1\textwidth}{}
      \plot{imgs-b_r-b05-000}{width=1\textwidth}{} 
 \end{columns}
\end{frame}
%-----------------------
 \begin{frame} 
 \begin{columns} 
    \column{0.5\textwidth}
      \plot{imgs-b_r-b05-050}{width=1\textwidth}{
        \klabellarge{20}{45}{Partial image}}
      \plot{wts-b05-050}{width=1\textwidth}{
        \klabellarge{20}{-10}{Source wavefield}}
    \column{0.5\textwidth}
      \plot{cross_corrs-b_r-b05-050}{width=1\textwidth}{
        \klabellarge{20}{45}{Wavefield product}}
      \plot{wtr-b05-050}{width=1\textwidth}{
        \klabellarge{20}{-10}{Receiver wavefield}}
 \end{columns}
\end{frame}
%-----------------------
 \begin{frame} 
 \begin{columns} 
    \column{0.5\textwidth}
      \plot{imgs-b_r-b05-100}{width=1\textwidth}{
        \klabellarge{20}{45}{Partial image}}
      \plot{wts-b05-100}{width=1\textwidth}{
        \klabellarge{20}{-10}{Source wavefield}}
    \column{0.5\textwidth}
      \plot{cross_corrs-b_r-b05-100}{width=1\textwidth}{
        \klabellarge{20}{45}{Wavefield product}}
      \plot{wtr-b05-100}{width=1\textwidth}{
        \klabellarge{20}{-10}{Receiver wavefield}}
 \end{columns}
\end{frame}
%-----------------------
 \begin{frame} 
 \begin{columns} 
    \column{0.5\textwidth}
      \plot{imgs-b_r-b05-150}{width=1\textwidth}{
        \klabellarge{20}{45}{Partial image}}
      \plot{wts-b05-150}{width=1\textwidth}{
        \klabellarge{20}{-10}{Source wavefield}}
    \column{0.5\textwidth}
      \plot{cross_corrs-b_r-b05-150}{width=1\textwidth}{
        \klabellarge{20}{45}{Wavefield product}}
      \plot{wtr-b05-150}{width=1\textwidth}{
        \klabellarge{20}{-10}{Receiver wavefield}}
  \end{columns}
\end{frame}
%-----------------------
 \begin{frame} 
 \begin{columns} 
    \column{0.5\textwidth}
      \plot{imgs-b_r-b05-200}{width=1\textwidth}{
        \klabellarge{20}{45}{Partial image}}
      \plot{wts-b05-200}{width=1\textwidth}{
        \klabellarge{20}{-10}{Source wavefield}}
    \column{0.5\textwidth}
      \plot{cross_corrs-b_r-b05-200}{width=1\textwidth}{
        \klabellarge{20}{45}{Wavefield product}}
      \plot{wtr-b05-200}{width=1\textwidth}{
        \klabellarge{20}{-10}{Receiver wavefield}}
 \end{columns}
\end{frame}
%-----------------------
 \begin{frame} 
 \begin{columns} 
    \column{0.5\textwidth}
      \plot{imgs-b_r-b05-250}{width=1\textwidth}{
        \klabellarge{20}{45}{Partial image}}
      \plot{wts-b05-250}{width=1\textwidth}{
        \klabellarge{20}{-10}{Source wavefield}}
    \column{0.5\textwidth}
      \plot{cross_corrs-b_r-b05-250}{width=1\textwidth}{
        \klabellarge{20}{45}{Wavefield product}}
      \plot{wtr-b05-250}{width=1\textwidth}{
        \klabellarge{20}{-10}{Receiver wavefield}}
 \end{columns}
\end{frame}
%-----------------------
 \begin{frame} 
 \begin{columns} 
    \column{0.5\textwidth}
      \plot{imgs-b_r-b05-300}{width=1\textwidth}{
        \klabellarge{20}{45}{Partial image}}
      \plot{wts-b05-300}{width=1\textwidth}{
        \klabellarge{20}{-10}{Source wavefield}}
    \column{0.5\textwidth}
      \plot{cross_corrs-b_r-b05-300}{width=1\textwidth}{
        \klabellarge{20}{45}{Wavefield product}}
      \plot{wtr-b05-300}{width=1\textwidth}{
        \klabellarge{20}{-10}{Receiver wavefield}}
 \end{columns}
\end{frame}
%-----------------------
 \begin{frame} 
 \begin{columns} 
    \column{0.5\textwidth}
      \plot{imgs-b_r-b05-350}{width=1\textwidth}{
        \klabellarge{20}{45}{Partial image}}
      \plot{wts-b05-350}{width=1\textwidth}{
        \klabellarge{20}{-10}{Source wavefield}}
    \column{0.5\textwidth}
      \plot{cross_corrs-b_r-b05-350}{width=1\textwidth}{
        \klabellarge{20}{45}{Wavefield product}}
      \plot{wtr-b05-350}{width=1\textwidth}{
        \klabellarge{20}{-10}{Receiver wavefield}}
 \end{columns}
\end{frame}
%-----------------------
 \begin{frame} 
 \begin{columns} 
    \column{0.5\textwidth}
      \plot{imgs-b_r-b05-400}{width=1\textwidth}{
        \klabellarge{20}{45}{Partial image}}
      \plot{wts-b05-400}{width=1\textwidth}{
        \klabellarge{20}{-10}{Source wavefield}}
    \column{0.5\textwidth}
      \plot{cross_corrs-b_r-b05-400}{width=1\textwidth}{
        \klabellarge{20}{45}{Wavefield product}}
      \plot{wtr-b05-400}{width=1\textwidth}{
        \klabellarge{20}{-10}{Receiver wavefield}}
 \end{columns}
\end{frame}
%-----------------------
 \begin{frame} 
 \begin{columns} 
    \column{0.5\textwidth}
      \plot{imgs-b_r-b05-450}{width=1\textwidth}{
        \klabellarge{20}{45}{Partial image}}
      \plot{wts-b05-450}{width=1\textwidth}{
        \klabellarge{20}{-10}{Source wavefield}}
    \column{0.5\textwidth}
      \plot{cross_corrs-b_r-b05-450}{width=1\textwidth}{
        \klabellarge{20}{45}{Wavefield product}}
      \plot{wtr-b05-450}{width=1\textwidth}{
        \klabellarge{20}{-10}{Receiver wavefield}}
 \end{columns}
\end{frame}
%-----------------------
 \begin{frame} 
 \begin{columns} 
    \column{0.5\textwidth}
      \plot{imgs-b_r-b05-500}{width=1\textwidth}{
        \klabellarge{20}{45}{Partial image}}
      \plot{wts-b05-500}{width=1\textwidth}{
        \klabellarge{20}{-10}{Source wavefield}}
    \column{0.5\textwidth}
      \plot{cross_corrs-b_r-b05-500}{width=1\textwidth}{
        \klabellarge{20}{45}{Wavefield product}}
      \plot{wtr-b05-500}{width=1\textwidth}{
        \klabellarge{20}{-10}{Receiver wavefield}}
 \end{columns}
\end{frame}
%-----------------------
 \begin{frame} 
 \begin{columns} 
    \column{0.5\textwidth}
      \plot{imgs-b_r-b05-550}{width=1\textwidth}{
        \klabellarge{20}{45}{Partial image}}
      \plot{wts-b05-550}{width=1\textwidth}{
        \klabellarge{20}{-10}{Source wavefield}}
    \column{0.5\textwidth}
      \plot{cross_corrs-b_r-b05-550}{width=1\textwidth}{
        \klabellarge{20}{45}{Wavefield product}}
      \plot{wtr-b05-550}{width=1\textwidth}{
        \klabellarge{20}{-10}{Receiver wavefield}}
 \end{columns}
\end{frame}
%-----------------------
 \begin{frame} 
 \begin{columns} 
    \column{0.5\textwidth}
      \plot{imgs-b_r-b05-600}{width=1\textwidth}{
        \klabellarge{20}{45}{Partial image}}
      \plot{wts-b05-600}{width=1\textwidth}{
        \klabellarge{20}{-10}{Source wavefield}}
    \column{0.5\textwidth}
      \plot{cross_corrs-b_r-b05-600}{width=1\textwidth}{
        \klabellarge{20}{45}{Wavefield product}}
      \plot{wtr-b05-600}{width=1\textwidth}{
        \klabellarge{20}{-10}{Receiver wavefield}}
 \end{columns}
\end{frame}

\begin{frame} \big{Fast velocity (+20\% error) movie} \end{frame}
%-----------------------
 \begin{frame} 
 \begin{columns} 
    \column{0.5\textwidth}
      \plot{imgs-b_r-b09-000}{width=1\textwidth}{}
      \plot{wts-b09-000}{width=1\textwidth}{}
    \column{0.5\textwidth}
      \plot{cross_corrs-b_r-b09-000}{width=1\textwidth}{}
      \plot{wtr-b09-000}{width=1\textwidth}{}
 \end{columns}
\end{frame}
%-----------------------
 \begin{frame} 
 \begin{columns} 
    \column{0.5\textwidth}
      \plot{imgs-b_r-b09-025}{width=1\textwidth}{}
      \plot{wts-b09-025}{width=1\textwidth}{}
    \column{0.5\textwidth}
      \plot{cross_corrs-b_r-b09-025}{width=1\textwidth}{}
      \plot{wtr-b09-025}{width=1\textwidth}{}
 \end{columns}
\end{frame}
%-----------------------
 \begin{frame} 
 \begin{columns} 
    \column{0.5\textwidth}
      \plot{imgs-b_r-b09-050}{width=1\textwidth}{}
      \plot{wts-b09-050}{width=1\textwidth}{}
    \column{0.5\textwidth}
      \plot{cross_corrs-b_r-b09-050}{width=1\textwidth}{}
      \plot{wtr-b09-050}{width=1\textwidth}{}
 \end{columns}
\end{frame}
%-----------------------
 \begin{frame} 
 \begin{columns} 
    \column{0.5\textwidth}
      \plot{imgs-b_r-b09-075}{width=1\textwidth}{}
      \plot{wts-b09-075}{width=1\textwidth}{}
    \column{0.5\textwidth}
      \plot{cross_corrs-b_r-b09-075}{width=1\textwidth}{}
      \plot{wtr-b09-075}{width=1\textwidth}{}
 \end{columns}
\end{frame}
%-----------------------
 \begin{frame} 
 \begin{columns} 
    \column{0.5\textwidth}
      \plot{imgs-b_r-b09-100}{width=1\textwidth}{}
      \plot{wts-b09-100}{width=1\textwidth}{}
    \column{0.5\textwidth}
      \plot{cross_corrs-b_r-b09-100}{width=1\textwidth}{}
      \plot{wtr-b09-100}{width=1\textwidth}{}
 \end{columns}
\end{frame}
%-----------------------
 \begin{frame} 
 \begin{columns} 
    \column{0.5\textwidth}
      \plot{imgs-b_r-b09-125}{width=1\textwidth}{}
      \plot{wts-b09-125}{width=1\textwidth}{}
    \column{0.5\textwidth}
      \plot{cross_corrs-b_r-b09-125}{width=1\textwidth}{}
      \plot{wtr-b09-125}{width=1\textwidth}{}
 \end{columns}
\end{frame}
%-----------------------
 \begin{frame} 
 \begin{columns} 
    \column{0.5\textwidth}
      \plot{imgs-b_r-b09-150}{width=1\textwidth}{}
      \plot{wts-b09-150}{width=1\textwidth}{}
    \column{0.5\textwidth}
      \plot{cross_corrs-b_r-b09-150}{width=1\textwidth}{}
      \plot{wtr-b09-150}{width=1\textwidth}{}
 \end{columns}
\end{frame}
%-----------------------
 \begin{frame} 
 \begin{columns} 
    \column{0.5\textwidth}
      \plot{imgs-b_r-b09-175}{width=1\textwidth}{}
      \plot{wts-b09-175}{width=1\textwidth}{}
    \column{0.5\textwidth}
      \plot{cross_corrs-b_r-b09-175}{width=1\textwidth}{}
      \plot{wtr-b09-175}{width=1\textwidth}{}
 \end{columns}
\end{frame}
%-----------------------
 \begin{frame} 
 \begin{columns} 
    \column{0.5\textwidth}
      \plot{imgs-b_r-b09-200}{width=1\textwidth}{}
      \plot{wts-b09-200}{width=1\textwidth}{}
    \column{0.5\textwidth}
      \plot{cross_corrs-b_r-b09-200}{width=1\textwidth}{}
      \plot{wtr-b09-200}{width=1\textwidth}{}
 \end{columns}
\end{frame}
%-----------------------
 \begin{frame} 
 \begin{columns} 
    \column{0.5\textwidth}
      \plot{imgs-b_r-b09-225}{width=1\textwidth}{}
      \plot{wts-b09-225}{width=1\textwidth}{}
    \column{0.5\textwidth}
      \plot{cross_corrs-b_r-b09-225}{width=1\textwidth}{}
      \plot{wtr-b09-225}{width=1\textwidth}{}
 \end{columns}
\end{frame}
%-----------------------
 \begin{frame} 
 \begin{columns} 
    \column{0.5\textwidth}
      \plot{imgs-b_r-b09-250}{width=1\textwidth}{}
      \plot{wts-b09-250}{width=1\textwidth}{}
    \column{0.5\textwidth}
      \plot{cross_corrs-b_r-b09-250}{width=1\textwidth}{}
      \plot{wtr-b09-250}{width=1\textwidth}{}
 \end{columns}
\end{frame}
%-----------------------
 \begin{frame} 
 \begin{columns} 
    \column{0.5\textwidth}
      \plot{imgs-b_r-b09-275}{width=1\textwidth}{}
      \plot{wts-b09-275}{width=1\textwidth}{}
    \column{0.5\textwidth}
      \plot{cross_corrs-b_r-b09-275}{width=1\textwidth}{}
      \plot{wtr-b09-275}{width=1\textwidth}{}
 \end{columns}
\end{frame}
%-----------------------
 \begin{frame} 
 \begin{columns} 
    \column{0.5\textwidth}
      \plot{imgs-b_r-b09-300}{width=1\textwidth}{}
      \plot{wts-b09-300}{width=1\textwidth}{}
    \column{0.5\textwidth}
      \plot{cross_corrs-b_r-b09-300}{width=1\textwidth}{}
      \plot{wtr-b09-300}{width=1\textwidth}{}
 \end{columns}
\end{frame}
%-----------------------
 \begin{frame} 
 \begin{columns} 
    \column{0.5\textwidth}
      \plot{imgs-b_r-b09-325}{width=1\textwidth}{}
      \plot{wts-b09-325}{width=1\textwidth}{}
    \column{0.5\textwidth}
      \plot{cross_corrs-b_r-b09-325}{width=1\textwidth}{}
      \plot{wtr-b09-325}{width=1\textwidth}{}
 \end{columns}
\end{frame}
%-----------------------
 \begin{frame} 
 \begin{columns} 
    \column{0.5\textwidth}
      \plot{imgs-b_r-b09-350}{width=1\textwidth}{}
      \plot{wts-b09-350}{width=1\textwidth}{}
    \column{0.5\textwidth}
      \plot{cross_corrs-b_r-b09-350}{width=1\textwidth}{}
      \plot{wtr-b09-350}{width=1\textwidth}{}
 \end{columns}
\end{frame}
%-----------------------
 \begin{frame} 
 \begin{columns} 
    \column{0.5\textwidth}
      \plot{imgs-b_r-b09-375}{width=1\textwidth}{}
      \plot{wts-b09-375}{width=1\textwidth}{}
    \column{0.5\textwidth}
      \plot{cross_corrs-b_r-b09-375}{width=1\textwidth}{}
      \plot{wtr-b09-375}{width=1\textwidth}{}
 \end{columns}
\end{frame}
%-----------------------
 \begin{frame} 
 \begin{columns} 
    \column{0.5\textwidth}
      \plot{imgs-b_r-b09-400}{width=1\textwidth}{}
      \plot{wts-b09-400}{width=1\textwidth}{}
    \column{0.5\textwidth}
      \plot{cross_corrs-b_r-b09-400}{width=1\textwidth}{}
      \plot{wtr-b09-400}{width=1\textwidth}{}
 \end{columns}
\end{frame}
%-----------------------
 \begin{frame} 
 \begin{columns} 
    \column{0.5\textwidth}
      \plot{imgs-b_r-b09-425}{width=1\textwidth}{}
      \plot{wts-b09-425}{width=1\textwidth}{}
    \column{0.5\textwidth}
      \plot{cross_corrs-b_r-b09-425}{width=1\textwidth}{}
      \plot{wtr-b09-425}{width=1\textwidth}{}
 \end{columns}
\end{frame}
%-----------------------
 \begin{frame} 
 \begin{columns} 
    \column{0.5\textwidth}
      \plot{imgs-b_r-b09-450}{width=1\textwidth}{}
      \plot{wts-b09-450}{width=1\textwidth}{}
    \column{0.5\textwidth}
      \plot{cross_corrs-b_r-b09-450}{width=1\textwidth}{}
      \plot{wtr-b09-450}{width=1\textwidth}{}
 \end{columns}
\end{frame}
%-----------------------
 \begin{frame} 
 \begin{columns} 
    \column{0.5\textwidth}
      \plot{imgs-b_r-b09-475}{width=1\textwidth}{}
      \plot{wts-b09-475}{width=1\textwidth}{}
    \column{0.5\textwidth}
      \plot{cross_corrs-b_r-b09-475}{width=1\textwidth}{}
      \plot{wtr-b09-475}{width=1\textwidth}{}
 \end{columns}
\end{frame}
%-----------------------
 \begin{frame} 
 \begin{columns} 
    \column{0.5\textwidth}
      \plot{imgs-b_r-b09-500}{width=1\textwidth}{}
      \plot{wts-b09-500}{width=1\textwidth}{}
    \column{0.5\textwidth}
      \plot{cross_corrs-b_r-b09-500}{width=1\textwidth}{}
      \plot{wtr-b09-500}{width=1\textwidth}{}
 \end{columns}
\end{frame}
%-----------------------
 \begin{frame} 
 \begin{columns} 
    \column{0.5\textwidth}
      \plot{imgs-b_r-b09-525}{width=1\textwidth}{}
      \plot{wts-b09-525}{width=1\textwidth}{}
    \column{0.5\textwidth}
      \plot{cross_corrs-b_r-b09-525}{width=1\textwidth}{}
      \plot{wtr-b09-525}{width=1\textwidth}{}
 \end{columns}
\end{frame}
%-----------------------
 \begin{frame} 
 \begin{columns} 
    \column{0.5\textwidth}
      \plot{imgs-b_r-b09-550}{width=1\textwidth}{}
      \plot{wts-b09-550}{width=1\textwidth}{}
    \column{0.5\textwidth}
      \plot{cross_corrs-b_r-b09-550}{width=1\textwidth}{}
      \plot{wtr-b09-550}{width=1\textwidth}{}
 \end{columns}
\end{frame}
%-----------------------
 \begin{frame} 
 \begin{columns} 
    \column{0.5\textwidth}
      \plot{imgs-b_r-b09-575}{width=1\textwidth}{}
      \plot{wts-b09-575}{width=1\textwidth}{}
    \column{0.5\textwidth}
      \plot{cross_corrs-b_r-b09-575}{width=1\textwidth}{}
      \plot{wtr-b09-575}{width=1\textwidth}{}
 \end{columns}
\end{frame}
%-----------------------
 \begin{frame} 
 \begin{columns} 
    \column{0.5\textwidth}
      \plot{imgs-b_r-b09-600}{width=1\textwidth}{}
      \plot{wts-b09-600}{width=1\textwidth}{}
    \column{0.5\textwidth}
      \plot{cross_corrs-b_r-b09-600}{width=1\textwidth}{}
      \plot{wtr-b09-600}{width=1\textwidth}{}
 \end{columns}
\end{frame}




\begin{frame} \frametitle{Defining the backscattered energy.}

\red{The only requirement to get it is to have a hard interface in the model}.

If that is the case, then, we can define:
\beq
\US= \US^b + \US^n,
\eeq

and
\beq
\UR= \UR^b + \UR^n.
\eeq

Using the conventional imaging condition ~\cite{claerbout:467}:
\beq
R(\bf{x})= \sum_{shots} \sum_t \US(\bf{x},t) \UR(\bf{x},t),
\label{eq:IC}
\eeq

Then,

\beq
R(\bf{x})^{}= R(\bf{x})^{nn} +R(\bf{x})^{nb}+R(\bf{x})^{bn}+ R(\bf{x})^{bb}.
\label{eq:cases}
\eeq

\end{frame}



\begin{frame}
We can generalize it using the extended imaging condition ~\cite{sava:S209}:
\beq
R(\bf{x_a},\bf{\lambda},\tau) =  \sum_{shots} \sum_t \US(\bf{x_a}-\bf{\lambda},t-\tau) \UR(\bf{x_a}+\bf{\lambda},t+\tau),
\eeq

and,

\beq
R(\bf{x_a},\bf{\lambda},\tau) = R(\bf{x_a},\bf{\lambda},\tau)^{nn} +R(\bf{x_a},\bf{\lambda},\tau)^{nb} +R(\bf{x_a},\bf{\lambda},\tau)^{bn} +R(\bf{x_a},\bf{\lambda},\tau)^{bb} 
\eeq
\end{frame}



\inputdir{flat}
\begin{frame} \frametitle{splitting the wavefield $\US$}

     $\US^{b}= \US - \US^{n}$               
    \plot{us_b05_ex}{width=0.5\textwidth}{  }
   
 \begin{columns} 
    \column{0.3\textwidth}
      \plot{wts-b05_ex}{width=1\textwidth}{}
    \column{0.3\textwidth}
      \plot{wts-n05_ex}{width=1\textwidth}{}
    \column{0.3\textwidth}
        \plot{mask05_ex}{width=1\textwidth}{}
 \end{columns}
 
 \end{frame}


 \begin{frame} 
 \plot{img05_ref}{height=0.45\textheight}{}
 \begin{columns} 
    \column{0.5\textwidth}
      \plot{citx05_ref}{height=0.45\textheight}{}
    \column{0.5\textwidth}
      \plot{cit05_ref}{height=0.45\textheight}{}
   \end{columns}
\end{frame}


\begin{frame}
 \plot{img05_ref}{height=0.3\textheight}{}
\begin{columns}
    \column{0.5\textwidth}
      \plot{img05_nn}{width=1\textwidth}{}
      \plot{img05_bn}{width=1\textwidth}{}
    \column{0.5\textwidth}
      \plot{img05_nb}{width=1\textwidth}{}
      \plot{img05_bb}{width=1\textwidth}{}
\end{columns}
\end{frame}


\begin{frame}
      \plot{citx05_ref}{height=0.45\textheight}{}
\begin{columns}
    \column{0.25\textwidth}
      \plot{citx05_nn}{height=0.35\textheight}{}
    \column{0.25\textwidth}
      \plot{citx05_nb}{height=0.35\textheight}{}
    \column{0.25\textwidth}
      \plot{citx05_bn}{height=0.35\textheight}{}
    \column{0.25\textwidth}
      \plot{citx05_bb}{height=0.35\textheight}{}

\end{columns}
\end{frame}

\begin{frame}
      \plot{cit05_ref}{height=0.45\textheight}{}
\begin{columns}
    \column{0.25\textwidth}
      \plot{cit05_nn}{height=0.35\textheight}{}
    \column{0.25\textwidth}
      \plot{cit05_nb}{height=0.35\textheight}{}
    \column{0.25\textwidth}
      \plot{cit05_bn}{height=0.35\textheight}{}
    \column{0.25\textwidth}
      \plot{cit05_bb}{height=0.35\textheight}{}

\end{columns}
\end{frame}







\begin{frame} \frametitle{Test experiments}
\begin{itemize}
   \item 11 velocity models: from 75\% to 125\%
   \item Surface receiver array.
   \item Tests with the hard interface in the velocity
\end{itemize}
\end{frame}
\cwpnote{}


\begin{frame} \frametitle{1D velocity profiles} \plot{vel1d}{height=0.8\textheight}{} \end{frame}





 \begin{frame} 
 \plot{img00_ref}{height=0.45\textheight}{}
 \begin{columns} 
    \column{0.4\textwidth}
      \plot{citx00_ref}{height=0.45\textheight}{}
    \column{0.4\textwidth}
      \plot{cit00_ref}{height=0.45\textheight}{}
   \column{0.2\textwidth}
      \plot{vel00-fat}{height=0.45\textheight}{}
   \end{columns}
\end{frame}
%-----------------------
 \begin{frame} 
 \plot{img01_ref}{height=0.45\textheight}{}
 \begin{columns} 
    \column{0.4\textwidth}
      \plot{citx01_ref}{height=0.45\textheight}{}
    \column{0.4\textwidth}
      \plot{cit01_ref}{height=0.45\textheight}{}
   \column{0.2\textwidth}
      \plot{vel01-fat}{height=0.45\textheight}{}
   \end{columns}
\end{frame}
%-----------------------
 \begin{frame} 
 \plot{img02_ref}{height=0.45\textheight}{}
 \begin{columns} 
    \column{0.4\textwidth}
      \plot{citx02_ref}{height=0.45\textheight}{}
    \column{0.4\textwidth}
      \plot{cit02_ref}{height=0.45\textheight}{}
   \column{0.2\textwidth}
      \plot{vel02-fat}{height=0.45\textheight}{}
   \end{columns}
\end{frame}
%-----------------------
 \begin{frame} 
 \plot{img03_ref}{height=0.45\textheight}{}
 \begin{columns} 
    \column{0.4\textwidth}
      \plot{citx03_ref}{height=0.45\textheight}{}
    \column{0.4\textwidth}
      \plot{cit03_ref}{height=0.45\textheight}{}
   \column{0.2\textwidth}
      \plot{vel03-fat}{height=0.45\textheight}{}
   \end{columns}
\end{frame}
%-----------------------
 \begin{frame} 
 \plot{img04_ref}{height=0.45\textheight}{}
 \begin{columns} 
    \column{0.4\textwidth}
      \plot{citx04_ref}{height=0.45\textheight}{}
    \column{0.4\textwidth}
      \plot{cit04_ref}{height=0.45\textheight}{}
   \column{0.2\textwidth}
      \plot{vel04-fat}{height=0.45\textheight}{}
   \end{columns}
\end{frame}
%-----------------------
 \begin{frame} 
 \plot{img05_ref}{height=0.45\textheight}{}
 \begin{columns} 
    \column{0.4\textwidth}
      \plot{citx05_ref}{height=0.45\textheight}{}
    \column{0.4\textwidth}
      \plot{cit05_ref}{height=0.45\textheight}{}
   \column{0.2\textwidth}
      \plot{vel05-fat}{height=0.45\textheight}{}
   \end{columns}
\end{frame}
%-----------------------
 \begin{frame} 
 \plot{img06_ref}{height=0.45\textheight}{}
 \begin{columns} 
    \column{0.4\textwidth}
      \plot{citx06_ref}{height=0.45\textheight}{}
    \column{0.4\textwidth}
      \plot{cit06_ref}{height=0.45\textheight}{}
   \column{0.2\textwidth}
      \plot{vel06-fat}{height=0.45\textheight}{}
   \end{columns}
\end{frame}
%-----------------------
 \begin{frame} 
 \plot{img07_ref}{height=0.45\textheight}{}
 \begin{columns} 
    \column{0.4\textwidth}
      \plot{citx07_ref}{height=0.45\textheight}{}
    \column{0.4\textwidth}
      \plot{cit07_ref}{height=0.45\textheight}{}
   \column{0.2\textwidth}
      \plot{vel07-fat}{height=0.45\textheight}{}
   \end{columns}
\end{frame}
%-----------------------
 \begin{frame} 
 \plot{img08_ref}{height=0.45\textheight}{}
 \begin{columns} 
    \column{0.4\textwidth}
      \plot{citx08_ref}{height=0.45\textheight}{}
    \column{0.4\textwidth}
      \plot{cit08_ref}{height=0.45\textheight}{}
   \column{0.2\textwidth}
      \plot{vel08-fat}{height=0.45\textheight}{}
   \end{columns}
\end{frame}
%-----------------------
 \begin{frame} 
 \plot{img09_ref}{height=0.45\textheight}{}
 \begin{columns} 
    \column{0.4\textwidth}
      \plot{citx09_ref}{height=0.45\textheight}{}
    \column{0.4\textwidth}
      \plot{cit09_ref}{height=0.45\textheight}{}
   \column{0.2\textwidth}
      \plot{vel09-fat}{height=0.45\textheight}{}
   \end{columns}
\end{frame}
%-----------------------
 \begin{frame} 
 \plot{img10_ref}{height=0.45\textheight}{}
 \begin{columns} 
    \column{0.4\textwidth}
      \plot{citx10_ref}{height=0.45\textheight}{}
    \column{0.4\textwidth}
      \plot{cit10_ref}{height=0.45\textheight}{}
   \column{0.2\textwidth}
      \plot{vel10-fat}{height=0.45\textheight}{}
   \end{columns}
\end{frame}



\begin{frame} \frametitle{Summary}
\begin{itemize}
    \item We now understand the backscattered energy.
    \item We understand which combinations contributes to the image.
    \item The backscattered energy  in the extended images demonstrated to be sensitive to velocity
    errors.
\end{itemize}
\end{frame}


\begin{frame} \frametitle{Acknowledgments}
I thank Paul and my classmates in Seismic Migration for great discussions!
\end{frame}




\bibliography{SEG}
