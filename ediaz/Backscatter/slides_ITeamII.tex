\bibliographystyle{seg}
\input{./Share/pcsmacros}

\bibliography{SEG}

\title[]{Sensitivity analysis of RTM backscattered energy for MVA}
\subtitle{}
\author[]{Esteban  D\'{i}az}
\institute{Center for Wave Phenomena \\
Colorado School of Mines \\
ediazpan@mines.edu
}
\date{}
\logo{}

\def\big#1{\begin{center} \LARGE \textbf{#1} \end{center}}
\def\cen#1{\begin{center}        \textbf{#1} \end{center}}

% ------------------------------------------------------------
\mode<beamer> { \cwpcover }

% ------------------------------------------------------------


\begin{frame}
Last time we saw:
	\begin{itemize}
		\item Construction of backscattered artifacts in the imaging process. 
		\item Identification in the image, and in the extended images.
		\item Noise or signal? 
	\end{itemize}
\end{frame}

\begin{frame}
What are we going to see now:
	\begin{itemize}
		\item Tests with two models.   
		\item Strategies to isolate the backscattered events without wavefield decomposition.
		\item Future research and tests.
	\end{itemize}
\end{frame}

\inputdir{flat}
\begin{frame} \big{Correct velocity movie} \end{frame}
%-----------------------
 \begin{frame} 
 \begin{columns} 
    \column{0.5\textwidth}
      \plot{wts-b05-000}{width=1\textwidth}{}
      \plot{wtr-b05-000}{width=1\textwidth}{}
      \plot{imgs-b_r-b05-000}{width=1\textwidth}{} 
 \end{columns}
\end{frame}
%-----------------------
 \begin{frame} 
 \begin{columns} 
    \column{0.5\textwidth}
      \plot{imgs-b_r-b05-050}{width=1\textwidth}{
        \klabellarge{20}{45}{Partial image}}
      \plot{wts-b05-050}{width=1\textwidth}{
        \klabellarge{20}{-10}{Source wavefield}}
    \column{0.5\textwidth}
      \plot{cross_corrs-b_r-b05-050}{width=1\textwidth}{
        \klabellarge{20}{45}{Wavefield product}}
      \plot{wtr-b05-050}{width=1\textwidth}{
        \klabellarge{20}{-10}{Receiver wavefield}}
 \end{columns}
\end{frame}
%-----------------------
 \begin{frame} 
 \begin{columns} 
    \column{0.5\textwidth}
      \plot{imgs-b_r-b05-100}{width=1\textwidth}{
        \klabellarge{20}{45}{Partial image}}
      \plot{wts-b05-100}{width=1\textwidth}{
        \klabellarge{20}{-10}{Source wavefield}}
    \column{0.5\textwidth}
      \plot{cross_corrs-b_r-b05-100}{width=1\textwidth}{
        \klabellarge{20}{45}{Wavefield product}}
      \plot{wtr-b05-100}{width=1\textwidth}{
        \klabellarge{20}{-10}{Receiver wavefield}}
 \end{columns}
\end{frame}
%-----------------------
 \begin{frame} 
 \begin{columns} 
    \column{0.5\textwidth}
      \plot{imgs-b_r-b05-150}{width=1\textwidth}{
        \klabellarge{20}{45}{Partial image}}
      \plot{wts-b05-150}{width=1\textwidth}{
        \klabellarge{20}{-10}{Source wavefield}}
    \column{0.5\textwidth}
      \plot{cross_corrs-b_r-b05-150}{width=1\textwidth}{
        \klabellarge{20}{45}{Wavefield product}}
      \plot{wtr-b05-150}{width=1\textwidth}{
        \klabellarge{20}{-10}{Receiver wavefield}}
  \end{columns}
\end{frame}
%-----------------------
 \begin{frame} 
 \begin{columns} 
    \column{0.5\textwidth}
      \plot{imgs-b_r-b05-200}{width=1\textwidth}{
        \klabellarge{20}{45}{Partial image}}
      \plot{wts-b05-200}{width=1\textwidth}{
        \klabellarge{20}{-10}{Source wavefield}}
    \column{0.5\textwidth}
      \plot{cross_corrs-b_r-b05-200}{width=1\textwidth}{
        \klabellarge{20}{45}{Wavefield product}}
      \plot{wtr-b05-200}{width=1\textwidth}{
        \klabellarge{20}{-10}{Receiver wavefield}}
 \end{columns}
\end{frame}
%-----------------------
 \begin{frame} 
 \begin{columns} 
    \column{0.5\textwidth}
      \plot{imgs-b_r-b05-250}{width=1\textwidth}{
        \klabellarge{20}{45}{Partial image}}
      \plot{wts-b05-250}{width=1\textwidth}{
        \klabellarge{20}{-10}{Source wavefield}}
    \column{0.5\textwidth}
      \plot{cross_corrs-b_r-b05-250}{width=1\textwidth}{
        \klabellarge{20}{45}{Wavefield product}}
      \plot{wtr-b05-250}{width=1\textwidth}{
        \klabellarge{20}{-10}{Receiver wavefield}}
 \end{columns}
\end{frame}
%-----------------------
 \begin{frame} 
 \begin{columns} 
    \column{0.5\textwidth}
      \plot{imgs-b_r-b05-300}{width=1\textwidth}{
        \klabellarge{20}{45}{Partial image}}
      \plot{wts-b05-300}{width=1\textwidth}{
        \klabellarge{20}{-10}{Source wavefield}}
    \column{0.5\textwidth}
      \plot{cross_corrs-b_r-b05-300}{width=1\textwidth}{
        \klabellarge{20}{45}{Wavefield product}}
      \plot{wtr-b05-300}{width=1\textwidth}{
        \klabellarge{20}{-10}{Receiver wavefield}}
 \end{columns}
\end{frame}
%-----------------------
 \begin{frame} 
 \begin{columns} 
    \column{0.5\textwidth}
      \plot{imgs-b_r-b05-350}{width=1\textwidth}{
        \klabellarge{20}{45}{Partial image}}
      \plot{wts-b05-350}{width=1\textwidth}{
        \klabellarge{20}{-10}{Source wavefield}}
    \column{0.5\textwidth}
      \plot{cross_corrs-b_r-b05-350}{width=1\textwidth}{
        \klabellarge{20}{45}{Wavefield product}}
      \plot{wtr-b05-350}{width=1\textwidth}{
        \klabellarge{20}{-10}{Receiver wavefield}}
 \end{columns}
\end{frame}
%-----------------------
 \begin{frame} 
 \begin{columns} 
    \column{0.5\textwidth}
      \plot{imgs-b_r-b05-400}{width=1\textwidth}{
        \klabellarge{20}{45}{Partial image}}
      \plot{wts-b05-400}{width=1\textwidth}{
        \klabellarge{20}{-10}{Source wavefield}}
    \column{0.5\textwidth}
      \plot{cross_corrs-b_r-b05-400}{width=1\textwidth}{
        \klabellarge{20}{45}{Wavefield product}}
      \plot{wtr-b05-400}{width=1\textwidth}{
        \klabellarge{20}{-10}{Receiver wavefield}}
 \end{columns}
\end{frame}
%-----------------------
 \begin{frame} 
 \begin{columns} 
    \column{0.5\textwidth}
      \plot{imgs-b_r-b05-450}{width=1\textwidth}{
        \klabellarge{20}{45}{Partial image}}
      \plot{wts-b05-450}{width=1\textwidth}{
        \klabellarge{20}{-10}{Source wavefield}}
    \column{0.5\textwidth}
      \plot{cross_corrs-b_r-b05-450}{width=1\textwidth}{
        \klabellarge{20}{45}{Wavefield product}}
      \plot{wtr-b05-450}{width=1\textwidth}{
        \klabellarge{20}{-10}{Receiver wavefield}}
 \end{columns}
\end{frame}
%-----------------------
 \begin{frame} 
 \begin{columns} 
    \column{0.5\textwidth}
      \plot{imgs-b_r-b05-500}{width=1\textwidth}{
        \klabellarge{20}{45}{Partial image}}
      \plot{wts-b05-500}{width=1\textwidth}{
        \klabellarge{20}{-10}{Source wavefield}}
    \column{0.5\textwidth}
      \plot{cross_corrs-b_r-b05-500}{width=1\textwidth}{
        \klabellarge{20}{45}{Wavefield product}}
      \plot{wtr-b05-500}{width=1\textwidth}{
        \klabellarge{20}{-10}{Receiver wavefield}}
 \end{columns}
\end{frame}
%-----------------------
 \begin{frame} 
 \begin{columns} 
    \column{0.5\textwidth}
      \plot{imgs-b_r-b05-550}{width=1\textwidth}{
        \klabellarge{20}{45}{Partial image}}
      \plot{wts-b05-550}{width=1\textwidth}{
        \klabellarge{20}{-10}{Source wavefield}}
    \column{0.5\textwidth}
      \plot{cross_corrs-b_r-b05-550}{width=1\textwidth}{
        \klabellarge{20}{45}{Wavefield product}}
      \plot{wtr-b05-550}{width=1\textwidth}{
        \klabellarge{20}{-10}{Receiver wavefield}}
 \end{columns}
\end{frame}
%-----------------------
 \begin{frame} 
 \begin{columns} 
    \column{0.5\textwidth}
      \plot{imgs-b_r-b05-600}{width=1\textwidth}{
        \klabellarge{20}{45}{Partial image}}
      \plot{wts-b05-600}{width=1\textwidth}{
        \klabellarge{20}{-10}{Source wavefield}}
    \column{0.5\textwidth}
      \plot{cross_corrs-b_r-b05-600}{width=1\textwidth}{
        \klabellarge{20}{45}{Wavefield product}}
      \plot{wtr-b05-600}{width=1\textwidth}{
        \klabellarge{20}{-10}{Receiver wavefield}}
 \end{columns}
\end{frame}




\begin{frame} \frametitle{Defining the backscattered energy.}

\red{The only requirement to get it is to have a hard interface in the model}.

If that is the case, then, we can define:
\beq
\US= \USr + \USt,
\eeq

and
\beq
\UR= \URr + \URt.
\eeq
Using the conventional imaging condition  ~\cite{claerbout:467}:
\beq
R(\xx)= \sum_{shots} \sum_t \US(\xx,t) \UR(\xx,t),
\label{eq:IC}
\eeq

Then,

\beq
R(\xx)= R^{tt}(\xx) +R^{tr}(\xx)+R^{rt}(\xx)+ R^{rr}(\xx).
\label{eq:cases}
\eeq

\end{frame}



\begin{frame}
We can generalize it using the extended imaging condition ~\cite{sava:S209}:
\beq
R({\bf c},\hh,\tau) =  \sum_{shots} \sum_t \US({\bf c}-\hh,t-\tau) \UR({\bf c}+\hh,t+\tau),
\eeq

and,

\beq
R({\bf c},\hh,\tau) = R^{tt}({\bf c},\hh,\tau) +R^{tr}({\bf c},\hh,\tau) +R^{rt}({\bf c},\hh,\tau) +R^{rr}({\bf c},\hh,\tau)^{rr} 
\eeq
\end{frame}


\inputdir{flat}
\begin{frame} \frametitle{splitting the wavefield $\US$}

     %$\US^{b}= \US - \US^{n}$               

    \plot{us_b05_ex}{width=0.5\textwidth}{
      \klabellarge{35}{+40}{$\US^{b} = $}  }
 
 \begin{columns} 
    \column{0.3\textwidth}
      \plot{wts-b05_ex}{width=1\textwidth}{
      \klabellarge{30}{-20}{$(\US  $}  
      \klabellarge{100}{-20}{$ -  $}  }
    \column{0.3\textwidth}
      \plot{wts-n05_ex}{width=1\textwidth}{
      \klabellarge{30}{-20}{$\USt ) $}  }
    \column{0.3\textwidth}
      \plot{mask05_ex}{width=1\textwidth}{
      \klabellarge{20}{-20}{$\times MASK$}  }
\end{columns}
 
 \end{frame}

 \begin{frame} 
 \plot{img05_ref}{height=0.45\textheight}{
 \klabellarge{47}{-5}{$R(\bf{x})$} }

 \begin{columns} 
    \column{0.5\textwidth}
      \plot{citx05_ref}{height=0.45\textheight}{
      \klabellarge{-40}{20}{$R(\bf{x},\bf{\lambda})$} }
    \column{0.5\textwidth}
      \plot{cit05_ref}{height=0.45\textheight}{
      \klabellarge{-40}{20}{$R(\bf{x},\bf{\tau})$} }
      
   \end{columns}
\end{frame}

\begin{frame}
 \plot{img05_ref}{height=0.3\textheight}{
 \wlabellarge{75}{3}{$R(\bf{x})$}}
\begin{columns}
    \column{0.5\textwidth}
      \plot{img05_nn}{width=1\textwidth}{
      \wlabellarge{70}{3}{$R(\bf{x})^{nn}$}}
      \plot{img05_bn}{width=1\textwidth}{
      \wlabellarge{70}{3}{$R(\bf{x})^{rt}$}}

    \column{0.5\textwidth}
      \plot{img05_nb}{width=1\textwidth}{
      \wlabellarge{70}{3}{$R(\bf{x})^{tr}$}}

      \plot{img05_bb}{width=1\textwidth}{
      \wlabellarge{70}{3}{$R(\bf{x})^{bb}$}}

\end{columns}
\end{frame}


\begin{frame}
      \plot{citx05_ref}{height=0.45\textheight}{
       \klabellarge{-50}{50}{$R(\bf{x},\bf{\lambda_x})$} }
\begin{columns}
    \column{0.25\textwidth}
      \plot{citx05_nn}{height=0.35\textheight}{
      \klabellarge{10}{105}{$R(\bf{x},\bf{\lambda_x})^{nn}$} } 
    \column{0.25\textwidth}
      \plot{citx05_nb}{height=0.35\textheight}{
      \klabellarge{10}{105}{$R(\bf{x},\bf{\lambda_x})^{tr}$} }
    \column{0.25\textwidth}
      \plot{citx05_bn}{height=0.35\textheight}{
      \klabellarge{10}{105}{$R(\bf{x},\bf{\lambda_x})^{rt}$} }
    \column{0.25\textwidth}
      \plot{citx05_bb}{height=0.35\textheight}{
      \klabellarge{10}{105}{$R(\bf{x},\bf{\lambda_x})^{bb}$} }

\end{columns}
\end{frame}

\begin{frame}
      \plot{cit05_ref}{height=0.45\textheight}{
       \klabellarge{-50}{50}{$R(\bf{x},\bf{\tau})$} }

\begin{columns}
    \column{0.25\textwidth}
      \plot{cit05_nn}{height=0.35\textheight}{
      \klabellarge{10}{105}{$R(\bf{x},\bf{\tau})^{nn}$} }
    \column{0.25\textwidth}
      \plot{cit05_nb}{height=0.35\textheight}{
      \klabellarge{10}{105}{$R(\bf{x},\bf{\tau})^{tr}$} }
    \column{0.25\textwidth}
      \plot{cit05_bn}{height=0.35\textheight}{
      \klabellarge{10}{105}{$R(\bf{x},\bf{\tau})^{rt}$} }

    \column{0.25\textwidth}
      \plot{cit05_bb}{height=0.35\textheight}{
      \klabellarge{10}{105}{$R(\bf{x},\bf{\tau})^{bb}$} }

\end{columns}
\end{frame}



\begin{frame} \frametitle{Test experiments}
\begin{itemize}
   \item 11 velocity models: from 85\% to 115\%
   \item Surface receiver array.
   \item Tests with the hard interface in the velocity
\end{itemize}
\end{frame}
\cwpnote{}


\begin{frame} \frametitle{1D velocity profiles} \plot{vel1d}{height=0.8\textheight}{} \end{frame}





 \begin{frame} 
 \plot{img00_ref}{height=0.45\textheight}{}
 \begin{columns} 
    \column{0.4\textwidth}
      \plot{citx00_back}{height=0.45\textheight}{}
    \column{0.4\textwidth}
      \plot{cit00_back}{height=0.45\textheight}{}
   \column{0.2\textwidth}
      \plot{vel00-fat}{height=0.45\textheight}{}
   \end{columns}
\end{frame}
%-----------------------
 \begin{frame} 
 \plot{img01_ref}{height=0.45\textheight}{}
 \begin{columns} 
    \column{0.4\textwidth}
      \plot{citx01_back}{height=0.45\textheight}{}
    \column{0.4\textwidth}
      \plot{cit01_back}{height=0.45\textheight}{}
   \column{0.2\textwidth}
      \plot{vel01-fat}{height=0.45\textheight}{}
   \end{columns}
\end{frame}
%-----------------------
 \begin{frame} 
 \plot{img02_ref}{height=0.45\textheight}{}
 \begin{columns} 
    \column{0.4\textwidth}
      \plot{citx02_back}{height=0.45\textheight}{}
    \column{0.4\textwidth}
      \plot{cit02_back}{height=0.45\textheight}{}
   \column{0.2\textwidth}
      \plot{vel02-fat}{height=0.45\textheight}{}
   \end{columns}
\end{frame}
%-----------------------
 \begin{frame} 
 \plot{img03_ref}{height=0.45\textheight}{}
 \begin{columns} 
    \column{0.4\textwidth}
      \plot{citx03_back}{height=0.45\textheight}{}
    \column{0.4\textwidth}
      \plot{cit03_back}{height=0.45\textheight}{}
   \column{0.2\textwidth}
      \plot{vel03-fat}{height=0.45\textheight}{}
   \end{columns}
\end{frame}
%-----------------------
 \begin{frame} 
 \plot{img04_ref}{height=0.45\textheight}{}
 \begin{columns} 
    \column{0.4\textwidth}
      \plot{citx04_back}{height=0.45\textheight}{}
    \column{0.4\textwidth}
      \plot{cit04_back}{height=0.45\textheight}{}
   \column{0.2\textwidth}
      \plot{vel04-fat}{height=0.45\textheight}{}
   \end{columns}
\end{frame}
%-----------------------
 \begin{frame} 
 \plot{img05_ref}{height=0.45\textheight}{}
 \begin{columns} 
    \column{0.4\textwidth}
      \plot{citx05_back}{height=0.45\textheight}{}
    \column{0.4\textwidth}
      \plot{cit05_back}{height=0.45\textheight}{}
   \column{0.2\textwidth}
      \plot{vel05-fat}{height=0.45\textheight}{}
   \end{columns}
\end{frame}
%-----------------------
 \begin{frame} 
 \plot{img06_ref}{height=0.45\textheight}{}
 \begin{columns} 
    \column{0.4\textwidth}
      \plot{citx06_back}{height=0.45\textheight}{}
    \column{0.4\textwidth}
      \plot{cit06_back}{height=0.45\textheight}{}
   \column{0.2\textwidth}
      \plot{vel06-fat}{height=0.45\textheight}{}
   \end{columns}
\end{frame}
%-----------------------
 \begin{frame} 
 \plot{img07_ref}{height=0.45\textheight}{}
 \begin{columns} 
    \column{0.4\textwidth}
      \plot{citx07_back}{height=0.45\textheight}{}
    \column{0.4\textwidth}
      \plot{cit07_back}{height=0.45\textheight}{}
   \column{0.2\textwidth}
      \plot{vel07-fat}{height=0.45\textheight}{}
   \end{columns}
\end{frame}
%-----------------------
 \begin{frame} 
 \plot{img08_ref}{height=0.45\textheight}{}
 \begin{columns} 
    \column{0.4\textwidth}
      \plot{citx08_back}{height=0.45\textheight}{}
    \column{0.4\textwidth}
      \plot{cit08_back}{height=0.45\textheight}{}
   \column{0.2\textwidth}
      \plot{vel08-fat}{height=0.45\textheight}{}
   \end{columns}
\end{frame}
%-----------------------
 \begin{frame} 
 \plot{img09_ref}{height=0.45\textheight}{}
 \begin{columns} 
    \column{0.4\textwidth}
      \plot{citx09_back}{height=0.45\textheight}{}
    \column{0.4\textwidth}
      \plot{cit09_back}{height=0.45\textheight}{}
   \column{0.2\textwidth}
      \plot{vel09-fat}{height=0.45\textheight}{}
   \end{columns}
\end{frame}
%-----------------------
 \begin{frame} 
 \plot{img10_ref}{height=0.45\textheight}{}
 \begin{columns} 
    \column{0.4\textwidth}
      \plot{citx10_back}{height=0.45\textheight}{}
    \column{0.4\textwidth}
      \plot{cit10_back}{height=0.45\textheight}{}
   \column{0.2\textwidth}
      \plot{vel10-fat}{height=0.45\textheight}{}
   \end{columns}
\end{frame}

\begin{frame}
We can define an DSO type objective function (OF) with time-lag gathers:
\beq
    J_{\tau}(s)= \frac{1}{2} \lnorm{P_{\tau}[R^{tr}(\xx,\tau)+R^{rt}(\xx,\tau)]}^2_2,
\eeq
or with space-lag gathers:

\beq
    J_{\hh}(s)= \frac{1}{2} \lnorm{P_{\hh}[R^{tr}(\xx,\hh)+R^{rt}(\xx,\hh)]}^2_2,
\eeq
or a joint OF:

\beq
    J(s)= J_{\tau}(s) + J_{\hh}(s) 
\eeq

where $P_\hh = |\hh|$, and $P_\tau=|\tau|$.

\end{frame}

\begin{frame}\frametitle{Time-lags OF}
\plot{OF_cit}{width=0.7\textwidth}{}
\end{frame}

\begin{frame}\frametitle{Space-lags OF}
\plot{OF_cix}{width=0.7\textwidth}{}
\end{frame}

\begin{frame}\frametitle{Joint OF}
\plot{OF_joint}{width=0.7\textwidth}{}
\end{frame}

\begin{frame}\frametitle{Lateral gradient model}
\begin{itemize}
    \item Two layers model
    \item Top layer has a lateral gradient
    \item A variable scalar error is applied to the top 
    layer. 
    \item The depth of the interface in the center 
    of the model remains the same.
    \item No wavefield decomposition.
\end{itemize}

\end{frame}


\inputdir{lateral}


\begin{frame} 
 \begin{columns} 
    \column{0.5\textwidth}
      \plot{Img01_ref}{width=1\textwidth}{}
      \plot{Cigx01_ref}{height=0.6\textheight}{}
    \column{0.5\textwidth}
      \plot{vel01}{width=1\textwidth}{}
      \plot{Cigt01_ref}{height=0.6\textheight}{}
 \end{columns}
\end{frame}


\begin{frame} 
 \begin{columns} 
    \column{0.5\textwidth}
      \plot{Img02_ref}{width=1\textwidth}{}
      \plot{Cigx02_ref}{height=0.6\textheight}{}
    \column{0.5\textwidth}
      \plot{vel02}{width=1\textwidth}{}
      \plot{Cigt02_ref}{height=0.6\textheight}{}
 \end{columns}
\end{frame}


\begin{frame} 
 \begin{columns} 
    \column{0.5\textwidth}
      \plot{Img03_ref}{width=1\textwidth}{}
      \plot{Cigx03_ref}{height=0.6\textheight}{}
    \column{0.5\textwidth}
      \plot{vel03}{width=1\textwidth}{}
      \plot{Cigt03_ref}{height=0.6\textheight}{}
 \end{columns}
\end{frame}


\begin{frame} 
 \begin{columns} 
    \column{0.5\textwidth}
      \plot{Img04_ref}{width=1\textwidth}{}
      \plot{Cigx04_ref}{height=0.6\textheight}{}
    \column{0.5\textwidth}
      \plot{vel04}{width=1\textwidth}{}
      \plot{Cigt04_ref}{height=0.6\textheight}{}
 \end{columns}
\end{frame}


\begin{frame} 
 \begin{columns} 
    \column{0.5\textwidth}
      \plot{Img05_ref}{width=1\textwidth}{}
      \plot{Cigx05_ref}{height=0.6\textheight}{}
    \column{0.5\textwidth}
      \plot{vel05}{width=1\textwidth}{}
      \plot{Cigt05_ref}{height=0.6\textheight}{}
 \end{columns}
\end{frame}


\begin{frame} 
 \begin{columns} 
    \column{0.5\textwidth}
      \plot{Img06_ref}{width=1\textwidth}{}
      \plot{Cigx06_ref}{height=0.6\textheight}{}
    \column{0.5\textwidth}
      \plot{vel06}{width=1\textwidth}{}
      \plot{Cigt06_ref}{height=0.6\textheight}{}
 \end{columns}
\end{frame}


\begin{frame} 
 \begin{columns} 
    \column{0.5\textwidth}
      \plot{Img07_ref}{width=1\textwidth}{}
      \plot{Cigx07_ref}{height=0.6\textheight}{}
    \column{0.5\textwidth}
      \plot{vel07}{width=1\textwidth}{}
      \plot{Cigt07_ref}{height=0.6\textheight}{}
 \end{columns}
\end{frame}


\begin{frame} 
 \begin{columns} 
    \column{0.5\textwidth}
      \plot{Img08_ref}{width=1\textwidth}{}
      \plot{Cigx08_ref}{height=0.6\textheight}{}
    \column{0.5\textwidth}
      \plot{vel08}{width=1\textwidth}{}
      \plot{Cigt08_ref}{height=0.6\textheight}{}
 \end{columns}
\end{frame}


\begin{frame} 
 \begin{columns} 
    \column{0.5\textwidth}
      \plot{Img09_ref}{width=1\textwidth}{}
      \plot{Cigx09_ref}{height=0.6\textheight}{}
    \column{0.5\textwidth}
      \plot{vel09}{width=1\textwidth}{}
      \plot{Cigt09_ref}{height=0.6\textheight}{}
 \end{columns}
\end{frame}


\begin{frame} 
 \begin{columns} 
    \column{0.5\textwidth}
      \plot{Img10_ref}{width=1\textwidth}{}
      \plot{Cigx10_ref}{height=0.6\textheight}{}
    \column{0.5\textwidth}
      \plot{vel10}{width=1\textwidth}{}
      \plot{Cigt10_ref}{height=0.6\textheight}{}
 \end{columns}
\end{frame}

\begin{frame}
Separating backscattered energy from primaries, what can we do?
\end{frame}


\begin{frame}
    \plot{Cigt05_ref}{height=0.8\textheight}{} 
\end{frame}

\begin{frame}
    \plot{Cig05KxKt}{height=0.8\textheight}{} 
\end{frame}

\begin{frame}
    \plot{Filt05}{height=0.8\textheight}{} 
\end{frame}

\begin{frame}
    \plot{Cigb05}{height=0.8\textheight}{} 
\end{frame}

\begin{frame}
    \plot{OFCigb}{height=0.8\textheight}{} 
\end{frame}


\begin{frame}
Why the minimum is not at model 5? 
    \begin{itemize}
        \item I realized yesterday that I did not record enough time
        \item This introduces edge effects.
        \item After correcting this ``acquisition" problem, EIC should be cleaner.
    \end{itemize}
\end{frame}



\begin{frame} \frametitle{Summary}
    \begin{itemize}
        \item The hard part is done. 
        \item Now comes the easy part.
        \item Tony's MVA framework could be adapted.
    \end{itemize}
\end{frame}

\begin{frame}
    What is next?
    \begin{itemize}
        \item Test on Sigsbee model.
        \item Theoretical framework for MVA.
        \item Suggestions are greatly appreciated!
    \end{itemize}
\end{frame}
\bibliography{SEG}
