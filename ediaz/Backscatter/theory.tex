\section{Theory}

The conventional imaging condition (I.C.) is a zero lag cross-correlation of the downgoing wavefield (source 
wavefield) with the upgoing wavefield (receiver wavefield). This supposition of ``downgoing" or ``upgoing"
is based on the single scattering consideration.

If the model we use for migration contains abrupt changes,  scattered energy will be generated by the interface,
therefore we will have part of the source or receiver wavefield going in the opposite direction as its expected 
by the conventional IC:


\beq
R(\bf{x})= \sum_{shots} \sum_t \US(\bf{x},t) \UR(\bf{x},t).
\label{eq:IC}
\eeq

This backscattered energy is undesired in the final image because it masks the geological information. When we have a 
hard interface the source wavefield has two components:
\beq
\US= \US^b + \US^n
\eeq
where $^b$ means backscattered energy and $^n$ means non-backscattered wavefield. The same logic applies to 
the receiver wave-field:
\beq
\UR= \UR^b + \UR^n.
\eeq

By introducing these new cases we can say that the RTM image is a linear combination between all the cases:

\beq
R(\bf{x})= R(\bf{x})^{nn} +R(\bf{x})^{nb}+R(\bf{x})^{bn}+ R(\bf{x})^{bb}.
\label{eq:cases}
\eeq

This analysis is similar as the one shown by ~\cite{fei:3130}.

Whatever combination but $\bf^{nn}$ will generate undesired incoherent low frequency energy in our image, and therefore
is considered as noise.

The main goal in imaging is to obtain a reliable earth model, the backscattered energy could carry some information to
help to constrain it. In order to investigate this velocity dependance I use extended images, introduced by ~\cite{sava:S209}.
These images are straight forward to calculate and contains important information about the model accuracy:

\beq
R(\bf{x_a},\bf{\lambda},\tau) =  \sum_{shots} \sum_t \US(\bf{x}-\bf{\lambda},t-\tau) \UR(\bf{x}+\bf{\lambda},t+\tau).
\eeq

To make feasible the velocity analysis we normally extract information in a subset of the image ($\bf{x_a} \in \bf{x}$). In this paper,
I focus my attention in time-lag gathers, therefore $\bf{\lambda}=0$.