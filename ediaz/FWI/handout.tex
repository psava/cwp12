\input{./Share/pcsmacros}

\author{Esteban D\'{i}az}
\title{Wave equation based refraction tomography}{An alternative to FWI beyond the cycle skipping}

% ------------------------------------------------------------

\section{introduction}
Diving wave energy is very important for the inversion goal in FWI. However, when the initial model is ``far" (we'll understand 
later what ``far" means) the inversion is destined to fail resulting in a model trapped in local minima.

Diving waves ( i.e. refracted) energy is necessary to contribute to the low wavenumber components of the model. Once the low wavenumber
model is achieved, then we can proceed to use other arrivals of the seismogram.





\section{MVA}

For diving waves we can penalize the energy outside zero $\tau$ lag, using the following function $P(\tau)=|\tau|$. By doing so, we can setup an 
inversion problem that minimizes the energy outside zero $\tau$ lag, by minimizing the OF:
\beq
J_\tau = \frac{1}{2} \norm{P(\tau)R(\xx,\tau)}^2_{\xx,\tau}
\label{eq:OF}
\eeq

Where, $R(\xx,\tau)$ is an extended image over the time lag $\tau$ defined as:

\beq
R(\xx,\tau) = \sum_{\bf e} \sum_t \US({\bf e},\xx,t -\tau)\UR({\bf e},\xx,t+\tau)
\label{eq:eic}
\eeq

For simplicity I will consider only one experiment, therefore from now on I eliminate the  sum over experiments $\bf e$.
To update the model we use the framework defined by the Adjoint State Method (ASM). 

\section{State variables}

The source and receiver wavefields (state variables) are generated using some sort of wave equation, in my case I will 
use the scalar (constant density) acoustic wave equation:

\beq
\swe{\US(\xx,t)}=f_s(\xx,t)
\label{eq:swe}
\eeq

One can simplify this equation, by defining the wave equation operator 
\beq
\mathcal{L}(\xx,t,{\bf m})=\swe{},
\eeq
then, equation~\ref{eq:swe} becomes:


\beq
\mathcal{L}(\xx,t,{\bf m})\US(\xx,t)=f_s(\xx,t)
\label{eq:us}
\eeq

For the receiver wavefield we apply the wave equation operator backward in time as we inject the data in the surface,
 therefore we apply the adjoint operator $\mathcal{L}^*$:
\beq
\mathcal{L}^*(\xx,t,{\bf m})\UR(\xx,t)=f_r(\xx,t)
\label{eq:ur}
\eeq


this means we will obtain as much sources as extended image locations we have.






We can rewrite equations~\ref{eq:us}and~\ref{eq:ur} in matrix form as follow:

\bea
\begin{bmatrix}
\mathcal{L}(\xx,t,{\bf m}) & 0 \\
0 & \mathcal{L}^*(\xx,t,{\bf m})
\end{bmatrix}
\begin{bmatrix}
\US(\xx,t) \\
\UR(\xx,t)
\end{bmatrix} =
\begin{bmatrix}
f_s(\xx,t)\\
f_r(\xx,t)
\end{bmatrix}
\label{eq:compact}
\eea


\section{Adjoint state variables}

To generate the adjoint variables (adjoint source and receiver wavefields) we need to create the adjoint source, which will
take us from the image to the wavefields, whereas in the state variables we go from the wavefields to the image. 

To generate the adjoint source we need to differentiate our OF respect the state variables, now I will show how we obtain
the adjoint source $g_s(\xx,t)$ for the $l^2$ norm:

Let's first start by expanding the objective function $J_\tau$ shown in equation~\ref{eq:OF}:
\[
J_\tau = \frac{1}{2} \norm{P(\tau)R(\xx,\tau)}^2_{\xx,\tau} = \frac{1}{2} \sum_{\xx,t}(P(\tau)R(\xx,\tau))^2
\]
then, by substituting equation~\ref{eq:eic} back in the OF:
\[
J_\tau= \frac{1}{2} \sum_{\xx,tau}\left (P(\tau)\sum_t \US(\xx,t -\tau)\UR(\xx,t+\tau) \right)^2
\]
now, 

\[
g_s(\xx,t)=\done{J_\tau}{\US} (\xx,t) = \sum_{\xx,tau}\left (P(\tau) \sum_t \US(\xx,t -\tau)\UR(\xx,t+\tau) \right) \left (P(\tau) \UR(\xx,t+\tau) \right),
\]
rearranging terms and substituting back the equation~\ref{eq:eic}:
\beq
g_s(\xx,t) = \sum_{\xx,tau}\left (P^2(\tau) R(\xx,\tau) \UR(\xx,t+\tau) \right)
\label{eq:adjs}
\eeq
by repeating the same process for the receiver side we have:
\[
g_r(\xx,t) = \sum_{\xx,tau}\left (P^2(\tau) R(\xx,\tau) \US(\xx,t-\tau) \right)
\label{eq:adjr}
\]


The sum over $\xx$ might be misleading, actually in this case the extended image locations acts as delta function $\delta(x)$,
for which we can apply the following property:

\beq
\int_{-\infty}^{\infty} f(x)\delta(x-c)dx = f(c),
\eeq
The adjoint sources are distributed along the extended images locations.

Once we have the adjoint sources, we proceed to obtain the adjoint state variables similarly as in equation~\ref{eq:compact}:
\bea
\begin{bmatrix}
\mathcal{L}(\xx,t,{\bf m}) & 0 \\
0 & \mathcal{L}^*(\xx,t,{\bf m})
\end{bmatrix}
\begin{bmatrix}
a_s(\xx,t) \\
a_r(\xx,t)
\end{bmatrix} =
\begin{bmatrix}
g_s(\xx,t)\\
g_r(\xx,t)
\end{bmatrix}
\eea
