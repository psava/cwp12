\input{./Share/pcsmacros}

\author{Esteban D\'{i}az}
\title{Multiscale seismic waveform inversion, by Bunks et al 1995}{summary}


% ------------------------------------------------------------

\section{introduction}

Bunks et al, makes a extensive review about the issues present in many seismic 
geophysical problems. The issues in this problems are directly correlated with the prescense
of local minima. 

To address the local minima problems, many people assume that the starting model 
is close to the correct answer, therefore one could linearize the problem to get rid,
in part, of the non-linearity. However, the power of resolution of such methods is not
optimal when the initial model is far from the global minimum.

The issue of local minima depends on the scale, therefore if one approach the model 
at different scale (starting from long wavelenght and refining the model towards short
wavelenght) components. 

Full Waveform Inversion is a highly non-linear problem, the objective function is usually
multi-modal. Therefore, depending on the starting model, the probability to fall
in a local minimum is very high. Bunks et al propose a multigrid method for different scales,
the methodology is implemented in the time-domain. On should first solve of the problem
by using first the low frequency components of the data, and then use the high frequency.
Such an approach have two main advantages: (1) in theory one could reach to a model
in the vicinity of global minimum if the frequency content of the data is low enough and (2)
solving for low frequencies is computationally much cheaper than for high frequencies.


