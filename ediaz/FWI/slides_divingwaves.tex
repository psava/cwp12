\bibliographystyle{seg}
\input{./Share/pcsmacros}
\pgfdeclareimage[height=0.75in]{mypic}{esteban}
\bibliography{SEG}

\title[]{Diving waves test}
\subtitle{}
\author[]{Esteban  D\'{i}az}
\institute{Center for Wave Phenomena \\
Colorado School of Mines }
\date{}
\logo{}

\def\big#1{\begin{center} \LARGE \textbf{#1} \end{center}}
\def\cen#1{\begin{center}        \textbf{#1} \end{center}}

% ------------------------------------------------------------
\mode<beamer> { \cwpcover }

% ------------------------------------------------------------
\begin{frame}

\end{frame}


\begin{frame}
Geometry is kept constant, velocity error is variable.\\
Images are gained independently thats why we see energy at 0\% error.
\end{frame}

\inputdir{ddiving}
\begin{frame}\plot{velS}{width=1\textwidth}{\klabellarge{50}{10}{Correct velocity}}\end{frame}
\begin{frame}\plot{dgrad-468-080}{width=1\textwidth}{\klabellarge{50}{10}{$-20\%$}}\end{frame}
\begin{frame}\plot{dgrad-468-090}{width=1\textwidth}{\klabellarge{50}{10}{$-10\%$}}\end{frame}
\begin{frame}\plot{dgrad-468-100}{width=1\textwidth}{\klabellarge{50}{10}{$0\%$}}\end{frame}
\begin{frame}\plot{dgrad-468-110}{width=1\textwidth}{\klabellarge{50}{10}{$10\%$}}\end{frame}
\begin{frame}\plot{dgrad-468-120}{width=1\textwidth}{\klabellarge{50}{10}{$20\%$}}\end{frame}


\begin{frame}\plot{Cit-468-080}{height=1\textheight}{\klabellarge{70}{10}{$-20\%$}}\end{frame}
\begin{frame}\plot{Cit-468-090}{height=1\textheight}{\klabellarge{70}{10}{$-10\%$}}\end{frame}
\begin{frame}\plot{Cit-468-100}{height=1\textheight}{\klabellarge{70}{10}{$0\%$}}\end{frame}
\begin{frame}\plot{Cit-468-110}{height=1\textheight}{\klabellarge{70}{10}{$10\%$}}\end{frame}
\begin{frame}\plot{Cit-468-120}{height=1\textheight}{\klabellarge{70}{10}{$20\%$}}\end{frame}

% ------------------------------------------------------------
\begin{frame}
Why do we see energy at $\tau=0$?\\
It is because of the adjoint source, the energy 
at zero lag comes from the data simulated with the wrong model.\\
Clearly, the velocity error introduces cycle skipping in the time-lags (two separated events). 
I think this could be optimized using DSO type of penalty, as we saw with the backscaterring. 
The problem seems to be similar to the backscattering energy optimization.
\end{frame}




\bibliography{SEG}
