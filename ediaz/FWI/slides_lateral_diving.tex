\bibliographystyle{seg}
\input{./Share/pcsmacros}
\pgfdeclareimage[height=0.75in]{mypic}{esteban}
\bibliography{SEG}

\title[]{Time-domain MVA in constant velocity}
\subtitle{a kernel analysis}
\author[]{Esteban  D\'{i}az}
\institute{Center for Wave Phenomena \\
Colorado School of Mines }
\date{}
\logo{}

\def\big#1{\begin{center} \LARGE \textbf{#1} \end{center}}
\def\cen#1{\begin{center}        \textbf{#1} \end{center}}

% ------------------------------------------------------------
\mode<beamer> { \cwpcover }

% ------------------------------------------------------------
\begin{frame}

\end{frame}


\begin{frame} \frametitle{experiment}
\begin{itemize}

\item Multiple sources, multiple receivers.
\item Constant velocity medium, with a $\pm30\%$ error
\end{itemize}
\end{frame}



\inputdir{lateral_cases}


\begin{frame}\frametitle{-20\% anomaly}
    \plot{vel-L}{width=\textwidth}{} 
\end{frame}

\begin{frame}\frametitle{0\% anomaly} 
    \plot{vel-C}{width=\textwidth}{} 
\end{frame}
\begin{frame}\frametitle{+20\% anomaly} 
    \plot{vel-H}{width=\textwidth}{} 
\end{frame}







\begin{frame}[fragile]{flow 1: Tony's mva}
    \begin{algorithm}[H]
        \begin{algorithmic}
            \FORALL{$\bf e$}
                \STATE obtain $u_s({\bf x},t,{\bf e})$, $u_r({\bf x},t,{\bf e})$
                \STATE $R(\xx,\tau,{\bf e}) \leftarrow u_s({\bf x},t-\tau,{\bf e})u_r({\bf x},t+\tau,{\bf e})$

            \ENDFOR
            \STATE $R(\xx,\tau)\leftarrow \sum_{\bf e}R(\xx,\tau,{\bf e})$
            \STATE $\tilde{R}(\xx,\tau) \leftarrow |\tau|R(\xx,\tau)$
            \FORALL{$\bf e$}
                \STATE obtain $g_s({\bf x},t,{\bf e})$, $g_r({\bf x},t,{\bf e})$ @ cip locations
                \STATE obtain $a_s({\bf x},t,{\bf e})$, $a_r({\bf x},t,{\bf e})$
                \STATE $g(\xx,{\bf e}) \leftarrow u_s({\bf x},t,{\bf e})a_s({\bf x},t,{\bf e})+u_r({\bf x},t,{\bf e})a_r({\bf x},t,{\bf e})$
            \ENDFOR
            \STATE  $g(\xx) \leftarrow \sum_{\bf e}g(\xx,{\bf e})$

        \end{algorithmic}
    \end{algorithm}
\end{frame}

\begin{frame}[fragile]{flow 2: Toto's mva}
    \begin{algorithm}[H]
        \begin{algorithmic}
            \FORALL{$\bf e$}
                \STATE obtain $u_s({\bf x},t,{\bf e})$, $u_r({\bf x},t,{\bf e})$
                \STATE $R(\xx,\tau,{\bf e}) \leftarrow u_s({\bf x},t-\tau,{\bf e})u_r({\bf x},t+\tau,{\bf e})$
                \STATE $\tilde{R}(\xx,\tau,{\bf e}) \leftarrow |\tau|R(\xx,\tau) I(\xx,{\bf e})$
                \STATE obtain $g_s({\bf x},t,{\bf e})$, $g_r({\bf x},t,{\bf e})$ @ cip locations
                \STATE obtain $a_s({\bf x},t,{\bf e})$, $a_r({\bf x},t,{\bf e})$
                \STATE $g(\xx,{\bf e}) \leftarrow u_s({\bf x},t,{\bf e})a_s({\bf x},t,{\bf e})+u_r({\bf x},t,{\bf e})a_r({\bf x},t,{\bf e})$
            \ENDFOR
            \STATE  $g(\xx) \leftarrow \sum_{\bf e}g(\xx,{\bf e})$

        \end{algorithmic}
    \end{algorithm}
    Where $I(\xx,{\bf e})$ is an illumination-based weighting function.
\end{frame}

\begin{frame} {$I(\xx)$}
The weighting function smoothly weights down parts
of the extended image which do not go through the 
wavepath between source and receivers.

\plot{img-c1-005-L}{width=\textwidth}{}

\end{frame}
\begin{frame} {$I(\xx)$}
The weighting function smoothly weights down parts
of the extended image which do not go through the 
wavepath between source and receivers.

\plot{img-c1-005-L-gg}{width=\textwidth}{}

\end{frame}


\begin{frame}
\begin{columns}
    \column{0.33\textwidth}
        \plot{eica-c2-005-L}{width=\textwidth}{\klabelnormal{30}{100}{$R_I(\xx,\tau)$}}
    \column{0.33\textwidth}
        \plot{mask_eica-c2-005-L}{width=\textwidth}{\klabelnormal{10}{100}{ $<env(R_I(\xx,\tau))>$}}
    \column{0.33\textwidth}
        \plot{I_mask_eica-c2-005-L}{width=\textwidth}{\klabelnormal{10}{100}{ $<env(R_I(\xx,0))>$}}
\end{columns}
\sep
$R_I$: extended illumination image (using modeled data as receiver wavefield).\\ 
$<>$: smoothing operator\\
$env()$: envelope
\end{frame}
\begin{frame}{Penalty function}
    \beq
        P(\xx,\tau)^2=\tau I({\xx}) {\bf e}^{-\tau^2/{s_\tau}^2} 
    \eeq
The exponential is a taper on the extended image, such that
the image at the last $\tau$ is approximately equal to zero.

\end{frame}


\begin{frame}
\begin{columns}
    \column{0.5\textwidth}
        \plot{eic-c2-L}{width=\textwidth}{\klabellarge{20}{100}{$R(\xx,\tau)$}}
    \column{0.5\textwidth}
        \plot{pen_eic-c2-L}{width=\textwidth}{\klabellarge{20}{100}{$P(\xx,\tau)R(\xx,\tau)$}}
\end{columns}
\end{frame}







\begin{frame}{QC design}

\begin{center}
    \begin{tabular}{ | c | c | c | c |}
    \hline
    Case   & sources & receivers & cip       \\ \hline
       0   &    1    &      1    &  1        \\ \hline
       1   &    1    &      1    & $nz$      \\ \hline
       2   &    1    &    $nr$   & $nz$      \\ \hline
       3   &   $ns$  &    $nr$   & $nz$      \\ \hline
       4   &   $ns$  &    $nr$   & $nz*ncig$ \\ \hline 
       5   &   $ns$  &    $nr$   & $nz*nx$   \\ 
    \hline
    \end{tabular}
\end{center}
\end{frame}
% ----------------------------------------------------------------------
% gradients and cases plots:
% ----------------------------------------------------------------------
\begin{frame}{Case 0}
    \plot{case-c0}{width=\textwidth}{}
\end{frame}

\begin{frame}{Case 0: -20\% gradient}
    \plot{grad-c0-L}{width=\textwidth}{}
\end{frame}
\begin{frame}{Case 0: 0\% gradient}
    \plot{grad-c0-C}{width=\textwidth}{}
\end{frame}
\begin{frame}{Case 0: +20\% gradient}
    \plot{grad-c0-H}{width=\textwidth}{}
\end{frame}

\begin{frame}{Case 1}
    \plot{case-c1}{width=\textwidth}{}
\end{frame}

\begin{frame}{Case 1: -20\% gradient}
    \plot{grad-c1-L}{width=\textwidth}{}
\end{frame}
\begin{frame}{Case 1: 0\% gradient}
    \plot{grad-c1-C}{width=\textwidth}{}
\end{frame}
\begin{frame}{Case 1: +20\% gradient}
    \plot{grad-c1-H}{width=\textwidth}{}
\end{frame}

\begin{frame}{Case 2}
    \plot{case-c2}{width=\textwidth}{}
\end{frame}

\begin{frame}{Case 2: -20\% gradient}
    \plot{grad-c2-L}{width=\textwidth}{}
\end{frame}
\begin{frame}{Case 2: 0\% gradient}
    \plot{grad-c2-C}{width=\textwidth}{}
\end{frame}
\begin{frame}{Case 2: +20\% gradient}
    \plot{grad-c2-H}{width=\textwidth}{}
\end{frame}


\begin{frame}{Case 3}
    \plot{case-c3}{width=\textwidth}{}
\end{frame}

\begin{frame}{Case 3: -20\% gradient}
    \plot{grad-c3-L}{width=\textwidth}{}
\end{frame}
\begin{frame}{Case 3: 0\% gradient}
    \plot{grad-c3-C}{width=\textwidth}{}
\end{frame}
\begin{frame}{Case 3: +20\% gradient}
    \plot{grad-c3-H}{width=\textwidth}{}
\end{frame}


\begin{frame}{Case 4}
    \plot{case-c4}{width=\textwidth}{}
\end{frame}

\begin{frame}{Case 4: -20\% gradient}
    \plot{grad-c4-L}{width=\textwidth}{}
\end{frame}
\begin{frame}{Case 4: 0\% gradient}
    \plot{grad-c4-C}{width=\textwidth}{}
\end{frame}
\begin{frame}{Case 4: +20\% gradient}
    \plot{grad-c4-H}{width=\textwidth}{}
\end{frame}


\begin{frame}{Case 5}
    \plot{case-c4}{width=\textwidth}{\klabellarge{20}{-5}{$\tau$ lag measures in all gridpoints}}
\end{frame}

\begin{frame}{Case 5: -20\% gradient}
    \plot{grad-c5-L}{width=\textwidth}{}
\end{frame}
\begin{frame}{Case 5: 0\% gradient}
    \plot{grad-c5-C}{width=\textwidth}{}
\end{frame}
\begin{frame}{Case 5: +20\% gradient}
    \plot{grad-c5-H}{width=\textwidth}{}
\end{frame}


% ----------------------------------------------------------------------
% end of gradients and cases plots:
% ----------------------------------------------------------------------



